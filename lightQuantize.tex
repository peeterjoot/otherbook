%
% Copyright � 2012 Peeter Joot.  All Rights Reserved.
% Licenced as described in the file LICENSE under the root directory of this GIT repository.
%

%
%
%\input{../peeter_prologue.tex}

%\chapter{Light and Electron Quantization?}
\label{chap:lightQuantize}
\date{July 2, 2009}
\revisionInfo{\(RCSfile: lightQuantize.tex,v \) Last \(Revision: 1.12 \) \(Date: 2009/12/03 03:24:40 \)}
\blogpage{http://sites.google.com/site/peeterjoot/math2009/lightQuantize.pdf}

%\newcommand{\Schrodinger}[0]{Schr\"{o}dinger}

\beginArtWithToc

\section{Motivation}

In \href{http://sites.google.com/site/peeterjoot/math2009/relwave.pdf}{Relativistic origins of Schr\"odinger's equation}, an ad-hoc technique of motivating the non-relativistic Schr\"odinger equation was detailed.  This used the DeBroglie quantization hypothesis to matter via a quantization of the Klein-Gordon equation.  While this  does result in the final desired goal some of the intermediate steps are rather unsatisfying.  In particular the wave equation for light in a vacuum was used as a starting point, before moving on to quantized matter.  Quantization of light itself was glossed over, and if that was considered in isolation, it is clear that photon quantization based on the wave equation in isolation cannot possibly be correct.  From the quantization result, whatever that is, one ought to be able to recover Maxwell's vacuum equation by considering time averages in a fashion something like the Ehrenfest procedure.  This cannot be done if the coupling between the field components is neglected.

A possible alternate starting point (simpler than the coupling between six field components) that would still use the wave equation would be to use Maxwell's equations in four vector potential form, however, the Lorentz gauge constraint would also be required

\begin{align}\label{eqn:maxwellPotVacuum}
\partial_\mu \partial^\mu A^\nu &= 0 \\
\partial_\nu A^\nu &= 0
\end{align}

In one fashion or another, one must have to start with the vacuum Maxwell's equation in its entirely if attempting light quantization starting from classical arguments.  Regardless of whether one considers the previous Schr\"odinger motivation attempt handwaving, the fact that it is possible at all points to a necessary failure of the Schr\"odinger equation if the wave function under consideration is for the electron.  Because the starting point for light quantization was wrong, the result for electron quantization must also be wrong.  With the coupling of light and electrons in classical electromagnetism, we need four coupled Schr\"odinger equations, not one.

A more correct looking treatment of photon quantization appears in \citep{de2004understanding}.  This paper utilizes an STA like \(F = \BE + i\BB\) notation, and Hermitian conjugation, but in a slightly different way.  In preparation for understanding that, here is an attempt to work through the classical Maxwell equation vacuum solutions for discrete frequencies.

\section{Maxwell vacuum solution}

\subsection{Fourier problem}

While Fourier solutions to \eqnref{eqn:maxwellPotVacuum} are possible, how does one deal with the coupling constraint of the Lorentz gauge?  Let us start over with the first order Maxwell equation, noting that \eqnref{eqn:maxwellPotVacuum} was the result of setting \(\grad \cdot A\) and \(J = 0\) in the general Maxwell equation

\EndArticle
