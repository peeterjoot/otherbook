%
% Copyright � 2012 Peeter Joot.  All Rights Reserved.
% Licenced as described in the file LICENSE under the root directory of this GIT repository.
%

%
%
%\input{../peeter_prologue.tex}

\newcommand{\e}[0]{\epsilon_0}

\chapter{Electrodynamic interaction energy and momentum}
\label{chap:interactionEnergy}
%\useCCL
\date{July 16, 2009}
\revisionInfo{\(RCSfile: interactionEnergy.tex,v \) Last \(Revision: 1.2 \) \(Date: 2009/12/03 03:24:40 \)}

\beginArtWithToc

\section{Motivation}

In \citep{GresnigtRelCl} examples of electromagnetic two particle interaction is considered to show equivalences between the Lorentz force equation and the Hamiltonian principle applied to Maxwell's equations.  Extend this to general two field superposition and see if the general Lorentz force (density) equation can be derived by considering application of the Hamiltonian principle to the stress energy tensor.

\section{Guts}

We wish to consider solutions to Maxwell's equation

\begin{equation}\label{eqn:interactionEnergy:20}
\begin{aligned}
\grad F = J/\e c
\end{aligned}
\end{equation}

Specifically, consider the superposition of two independent solutions \(F_1\), and \(F_2\) of Maxwell's equation

\begin{equation}\label{eqn:interactionEnergy:40}
\begin{aligned}
\grad F_k = J_k/\e c
\end{aligned}
\end{equation}

If the energy of the fields can be used to compute the interaction, it is relativistically more natural seeming to employ the
energy momentum tensor

\begin{equation}\label{eqn:interactionEnergy:60}
\begin{aligned}
T(a) &= -\frac{\e}{2} F a F
\end{aligned}
\end{equation}

For the superposition field \(F = F_1 + F_2\) this is

\begin{equation}\label{eqn:interactionEnergy:80}
\begin{aligned}
T(a) &= -\frac{\e}{2} (F_1 a F_1 + F_2 a F_2 + F_2 a F_1 + F_1 a F_2)
\end{aligned}
\end{equation}

The divergence of the energy momentum tensor can be related to the currents
\begin{equation}\label{eqn:interactionEnergy:100}
\begin{aligned}
\grad \cdot T(a)
&= -\frac{\e}{2} \gpgradezero{\grad ((F_1 + F_2)a(F_1 + F_2))} \\
&= -\frac{\e}{2} \gpgradeone{ F_1 \lrgrad F_1 + F_2 \lrgrad F_2  + F_1 \lrgrad F_2  + F_2 \lrgrad F_1 } \cdot a \\
\end{aligned}
\end{equation}

Cyclic reordering of the scalar product factors has been used to factor out the \(a\).  Now, consider one of the products

\begin{equation}\label{eqn:interactionEnergy:120}
\begin{aligned}
F_k \lrgrad F_j &= F_k (\rgrad F_j) + (F_k \lgrad) F_j
\end{aligned}
\end{equation}

and observe that the reverse of Maxwell's equation is \((\rgrad F)^{\tilde{}} = J/\e c = - (F \lgrad)\), so we have

\begin{equation}\label{eqn:interactionEnergy:140}
\begin{aligned}
\gpgradeone{ F_k \lrgrad F_j }
&= \gpgradeone{ F_k J_j - J_k F_j} \\
&= 2 F_k \cdot J_j
\end{aligned}
\end{equation}

This supplies a compact expression for the superposition stress energy tensor's divergence

\begin{equation}\label{eqn:interactionEnergy:160}
\begin{aligned}
\grad \cdot T(a)
&= -{\e} \left( F_1 \cdot J_1 + F_2 \cdot J_2  + F_1 \cdot J_2  + F_2 \cdot J_1 \right) \cdot a
\end{aligned}
\end{equation}

\EndArticle
