%
% Copyright � 2012 Peeter Joot.  All Rights Reserved.
% Licenced as described in the file LICENSE under the root directory of this GIT repository.
%

%
%
%\documentclass{article}      % Specifies the document class

%\input{../peeters_macros.tex}

%
% The real thing:
%

%\usepackage[bookmarks=true]{hyperref}

                             % The preamble begins here.
\chapter{sr lagrangian}
\label{chap:srLagrangianQ}
%\title{} % Declares the document's title.
%\author{Peeter Joot \quad peeterjoot@protonmail.com}         % Declares the author's name.
\date{ \(RCSfile: srLagrangianQ.tex,v \) Last \(Revision: 1.8 \) \(Date: 2009/06/14 23:51:45 \) }

%\begin{document}             % End of preamble and beginning of text.

%\maketitle{}

%\tableofcontents

%\section{}

Yes, \(A\) is a vector, and the spacetime basis is \(\{\gamma_\mu\}\).

Coordinates representation is done by dotting with the reciprocal frame
vectors satisfying the following:

\begin{equation}\label{eqn:srLagrangianQ:20}
\begin{aligned}
\gamma_\mu \cdot \gamma^\nu = {\delta^\nu}_\mu
\end{aligned}
\end{equation}

For example for \(A\)

\begin{equation}\label{eqn:srLagrangianQ:40}
\begin{aligned}
A &= a^\alpha \gamma_\alpha \\
A \cdot \gamma^\beta
&= a^\alpha \gamma_\alpha \cdot \gamma^\beta \\
&= a^\alpha {\delta_\alpha}^\beta \\
&= a^\beta \\
\implies \\
A &= \left(A \cdot \gamma^\alpha\right) \gamma_\alpha \equiv A^\alpha \gamma_\alpha \\
\end{aligned}
\end{equation}

Similarly, computation of the coordinates in terms of the reciprocal frame
one has

\begin{equation}\label{eqn:srLagrangianQ:60}
\begin{aligned}
A &= \left(A \cdot \gamma_\alpha\right) \gamma^\alpha \equiv A_\alpha \gamma^\alpha \\
\end{aligned}
\end{equation}

%My view of the tensor form is generally what you get when you drop the basis, looking only at the coordinates.  I have found
%that there is often reasons to temporarily introduce tensor representations, but that they can often be temporary.  Some of the
%calculations I have attempted can probably be done with non-tensor methods, but they are not too bad once you get used to them.
Using the above coordinate representation \(A \cdot \gamma_\mu\) (yes, it is a scalar), can be observed to be:

\begin{equation}\label{eqn:srLagrangianQ:80}
\begin{aligned}
A \cdot \gamma_\mu
&= \left( A_\nu \gamma^\nu \right) \cdot \gamma_\mu \\
&= A_\nu {\delta^\nu}_\mu \\
&= A_\mu
\end{aligned}
\end{equation}

I see that I assumed \(v\) was understood to be:

\begin{equation}\label{eqn:srLagrangianQ:100}
\begin{aligned}
x &= x^\mu \gamma_\mu \\
v &= \frac{dx}{d\tau} = \xdot^\mu \gamma_\mu
\end{aligned}
\end{equation}

(that is one of the missing steps to get from \(\PD{\xdot^\mu}{A \cdot v}\) to \(A \cdot \gamma_\mu\)).

Perhaps this makes more sense:

\begin{equation}\label{eqn:srLagrangianQ:120}
\begin{aligned}
%\frac{d}{d\tau}
\PD{\xdot^\mu}{A \cdot v}
&= \PD{\xdot^\mu}{} A \cdot \xdot^\nu \gamma_\nu \\
&= A \left( \PD{\xdot^\mu}{\xdot^\nu} \right) \cdot \gamma_\nu \\
&= {\delta^\mu}_\nu A \cdot \gamma_\nu \\
&= A \cdot \gamma_\mu \\
\end{aligned}
\end{equation}

so for the derivative

\begin{equation}\label{eqn:srLagrangianQ:140}
\begin{aligned}
\frac{d}{d\tau} \PD{\xdot^\mu}{A \cdot v}
&= \frac{d}{d\tau} A \cdot \gamma_\mu \\
&= \frac{d}{d\tau} (A^\nu \gamma_\nu) \cdot \gamma_\mu \\
&= \frac{dA^\nu}{d\tau} \gamma_\nu \cdot \gamma_\mu \\
\end{aligned}
\end{equation}

You included a nice little box summary of Maxwell's equation in tensor form.  That is pretty much identical to the curl
form, but to see the equivalence the product operations on a wedge are required.  Before mentioning any of that does this
part above now make sense?  I will revise my notes to try to clarify it.

%%\bibliographystyle{plain}
%\bibliographystyle{plainnat} % supposed to allow for \url use.
%\bibliography{myrefs}      % expects file "myrefs.bib"

%\end{document}               % End of document.
