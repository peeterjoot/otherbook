%
% Copyright � 2012 Peeter Joot.  All Rights Reserved.
% Licenced as described in the file LICENSE under the root directory of this GIT repository.
%

%
%
%\documentclass{article}      % Specifies the document class

%\input{../peeters_macros.tex}

%
% The real thing:
%

%\usepackage[bookmarks=true]{hyperref}

                             % The preamble begins here.
\chapter{Some GR Notes} % Declares the document's title.
\label{chap:schwartzchildMetric}
%\author{Peeter Joot \quad peeterjoot@protonmail.com}         % Declares the author's name.
\date{ October 2, 2008.  \(RCSfile: schwartzchildMetric.tex,v \) Last \(Revision: 1.20 \) \(Date: 2009/06/14 23:51:45 \) }

%\begin{document}             % End of preamble and beginning of text.

%\maketitle{}

%\tableofcontents

\section{Motivation}

Some General relativity notes exploring ideas from emails with Lut Mentz.
Prior to this I was under the impression that I had zero knowledge of GR,
but it turns out that many of the ideas are really action based.
Allowing the spacetime unit vectors to vary, something we are free to
do in SR or Newtonian mechanics too, results in a more
general metric in a purely kinetic Lagrangian.  This metric variation
can be interpreted as a mechanism for introducing more general
accelerations very similar to fictitious forces that one sees in a
rotating frame or other non-uniform coordinate system.

These notes contain my attempt to walk through some of these ideas, to see
if I can coherently explain them to myself.  If I can not do so then I do not
understand things sufficiently.  Being able to produce such an explanation
may not mean that I truly understand the issues, but it is a required
first step.

Useful references are Lut's writeup \citep{lutSchwarzChildRadial},
and the Schwarzchild calculation \citep{mathpagesSchwarzChildRadial}
from the online text reflections on relativity.

FIXME: Update here when Lut puts his Rindler metric derivation online.

First emails with Lut was about
\href{http://www.physicsforums.com/showthread.php?t=252187}{Lagrangian mass variation}, where
he was investigating the similarities between varying mass directly and the spatial/metric
variation of GR.  Some of the math bits of this can be found in \citep{gabookII:PJMassVary}, but the
exploratory physics bits he was attempting are probably more interesting.

\subsection{Lagrangian for General Relativity}

The equations of motion resulting from a purely kinetic Lagrangian

\begin{equation}\label{eqn:schwartzchild_metric:keLagrangian}
\LL = \inv{2} \sum  g_{b c}(q^a) \qdot^b \qdot^c,
\end{equation}

can be found to be

\begin{equation}\label{eqn:schwartzchild_metric:geodesics}
\begin{aligned}
0
&= \qddot^a + \qdot^b \qdot^c {\Gamma^a}_{b c} \\
{\Gamma^a}_{b c} &= \inv{2} g^{a d} \left(
\PD{q^b}{g_{c d}}
+ \PD{q^c}{g_{d b}}
- \PD{q^d}{g_{b c}}
\right)
\end{aligned}
\end{equation}

One such derivation can be found in
the solution of problem one \citep{classicalmechanics:PJTongMf1}, associated with the
Lagrangian problem set for
Dr. David Tong's online mechanics text \citep{TongDynamics}.

This is the Lagrangian for general relativity, once the metric tensor \(g_{a b}\) is specified.

\subsection{General kinetic Lagrangian for fixed frame vectors}

One does not have to go to GR to find Kinetic energy expressions of the form in \eqnref{eqn:schwartzchild_metric:keLagrangian}.

A simple example of a more general Kinetic energy description can be found by any use of non-orthonormal basis vectors, say \(\{\Be_i\}\),
for the space.

Given such a non-orthonormal frame, the trick to calculating the coordinates is tied to an alternate set of basis vectors, called the reciprocal frame.
Provided the initial set of vectors spans the space, one can always calculate this second pair such that they meet the following relationships:

\begin{equation*}
\Be^i \cdot \Be_j = {\delta^i}_j
\end{equation*}

Calculating these reciprocal frame vectors is a linear algebra problem, essentially requiring a matrix inversion.  Here we just assume they can be calculated and use them as a convenient way to find the coordinates in this non-orthogonal frame

\begin{equation}\label{eqn:schwartzchildMetric:20}
\begin{aligned}
\Bx &= \sum \Be_j a_j \\
\Bx \cdot \Be^i &= \sum (\Be_j a_j) \cdot \Be^i = \sum {\delta^j}_i a_j = a_i \\
\implies \\
\Bx &= \sum \Be_j (\Bx \cdot \Be^j)
\end{aligned}
\end{equation}

It is customary to write \(a_i = \Bx \cdot \Be^i = x^i\), in order to have
mixed upper and lower indices for implied summation.

\begin{equation*}
\Bx = \sum \Be_j x^j = \Be_j x^j
\end{equation*}

Once one has a way of calculating coordinates for an arbitrary basis, other quantities such as velocity
can then be calculated

\begin{equation*}
\Bv^2 = \dot{\Bx} \cdot \dot{\Bx} = (\Be_j \cdot \Be_k) \xdot^j \xdot^k
\end{equation*}

This
\(\Be_j \cdot \Be_k\) coefficient of the coordinates
gets a special name, the metric tensor
\(g_{j k} = \Be_j \cdot \Be_k\).  It is a symmetric and invertible quantity and can be employed to
express the Kinetic Energy term of a single particle Lagrangian in the general tensor form

\begin{equation*}
\inv{2} m \Bv^2 = \inv{2} m g_{j k} \xdot^j \xdot^k
\end{equation*}

Now, in this case \(g_{j k}\) is not a function of the coordinates (ie: of position) as in
\eqnref{eqn:schwartzchild_metric:keLagrangian}.

\subsection{General kinetic expression for moving frame vectors}

Now, in the GR case the metric varies (or may) with position.  What do we need to observe this same form in Newtonian
physics?  The immediate thing that comes to mind is the use of a curvilinear basis, where the basis vectors
for a space are allowed to vary direction with position along some path.  Initially that seemed reasonable to
me but for some arbitrary parametrized path, would not the metric then also vary with the path parametrization?  If that
was the case, the Lagrangian in \eqnref{eqn:schwartzchild_metric:keLagrangian} does not have the form of a kinetic energy expression.

To resolve this I considered an example.  I can lay out two directions in my backyard, one along the vegetable garden
parallel to the house roughly pointing north, and another diagonally across to my gate.  This logically defines a coordinate
system or set of frame vectors that I can make local measurements with respect to.
Now, translation of this coordinate system
to my dad's house 40 km to the south will not be a particularly logical for measuring there.  He lives on a very
steep hill.

I can pace out distances in my backyard without having to consider the curvature
of the Earth and my dad can do the same for a long stretch of the hill walking up the street towards his house.
The local frame vectors can be considered to lie along a flat surface if that surface area is small enough.
Alternately we can say the associated metric associated with a surface coordinate system for a point on surface of the Earth can be considered constant for small enough measurements.

Now, to express the same ideas mathematically, consider a curve expressed parametrically between two points, such that all
points along the path take the values

\begin{equation*}
\Bx(\lambda) = x^i(\lambda) \Be_i(\Bx)
\end{equation*}

The vector distance between two points on this path is

\begin{equation*}
\Bx(\lambda_2) - \Bx(\lambda_1) = x^i(\lambda_2) \Be_i(\Bx(\lambda_2)) - x^i(\lambda_1) \Be_i(\Bx(\lambda_1))
\end{equation*}

but this is the direct difference in position between these two points, not the distance along the curve.
To be a true measure of the distance the difference in position has to also be small enough that the frame vectors lie in the same direction at both points to some approximation.

Given such an approximation one can then write
\begin{equation}\label{eqn:schwartzchildMetric:40}
\begin{aligned}
\Bx(\lambda_2) - \Bx(\lambda_1) &= \left(x^i(\lambda_2) - x^i(\lambda_1)\right) \Be_i(\Bx) \\
d\Bx &= \frac{d x^i}{d\lambda} \Be_i(\Bx) d\lambda \\
\end{aligned}
\end{equation}

For such a representation to be valid, the variation of \(\Be_i\) at the point \(\Bx\) has to be small enough that \(\Be_i\) can be
considered constant.  This is still not a very well defined statement mathematically, and it is not too hard to imagine
scenarios where it totally fails.  An example is a fractal like curve, something continuous but not differentiable at
any point.

Assuming a sufficiently differentiable curve then the distance along the curve between two points can be obtained from the
integral

\begin{equation}\label{eqn:schwartzchildMetric:60}
\begin{aligned}
ds
&= \int_{\lambda_1}^{\lambda_2} \sqrt{\left(\frac{d\Bx}{d\lambda}\right)^2} d\lambda \\
&= \int_{\lambda_1}^{\lambda_2} \sqrt{ g_{ i j } \frac{dx^i}{d\lambda} \frac{dx^j}{d\lambda} } d\lambda \\
\end{aligned}
\end{equation}

Or
\begin{equation}\label{eqn:schwartzchildMetric:80}
\begin{aligned}
\left(\frac{ds}{d\lambda}\right)^2 = g_{ i j } \frac{dx^i}{d\lambda} \frac{dx^j}{d\lambda}
\end{aligned}
\end{equation}

Now, what is the most natural parametrization?  For physical situations time comes to mind, but if the particle stops for a
while on the path, then this derivative goes to zero for a while even if the curve is continuous and has derivatives of all
orders at all points.  Falling back to the most simple curve as a motivator, the circle,
use of fractions \(\theta\) of the total circumference of the circle \(2\pi\) naturally parametrizes points on the curve.  The same thing can be done for any
curve in Euclidean space, using arc length to parametrize a path.  In terms
of the metric tensor this is

\begin{equation}\label{eqn:schwartzchildMetric:100}
\begin{aligned}
\left(\frac{ds}{ds}\right)^2 = 1 = g_{ i j } \frac{dx^i}{ds} \frac{dx^j}{ds}
\end{aligned}
\end{equation}

Introducing proper time for spacetime event path parametrization will
be related to these spatial ideas.  Before considering proper time some
thought about general metrics for spacetime seems to be in order.

\section{Metrics for spacetime in special relativity}

It was stated above that \eqnref{eqn:schwartzchild_metric:keLagrangian} was the
Lagrangian for general relativity, once the metric is specified.

It is not
at all obvious to me what physically motivates the choice of metric.
In the two examples that Lut has given me, the Rindler metric and the
Schwartzchild metric, this appears to be rather arbitrary.

\subsection{Minkowski metric justification from the wave equation}

Picking the
Minkowski metric for special relativity is not at all an obvious
choice.  An introductory physics book like \citep{lewis1965mbp} chooses
to introduce the idea indirectly by requiring that a coordinate transformation
used to express the equation for a spherically expanding light shell

\begin{equation}\label{eqn:schwartzchild_metric:shell}
\begin{aligned}
\sum_i (x^i)^2 - c^2 t^2 = \sum_i ({x^i}')^2 - c^2 {t'}^2
\end{aligned}
\end{equation}

is identical in any spacetime coordinate system.  This expresses the idea that the speed
of light is the same in any frame of reference and that both time and distance
are to be measured only locally.  The Lorentz transformation can then be derived from
this notion of spherical shell equation invariance, and the Minkowski mixed signature
metric can be observed to fundamentally describe both the Lorentz transformation and the
shell invariance.

First reading this it was not at all obvious to me that \eqnref{eqn:schwartzchild_metric:shell} was a reasonable starting
point.  The constancy of the speed of light is easy to say, but saying it does not mean
that the implications are understood.

Suppose that one is considering the standard relativistic scenario of a fast spaceship (\(v = c/2\) say),
with headlights on.  Is it that obvious that the speed of light perceived by an observer of the rocket
will be \(c\), or will it be \(1.5 c\)?

This non-obvious nature is likely reflected by popular disbelief of relativity in the general non-physics student/professional population.
The twin paradox, length contraction, and time dilatation ideas are all very far from our general
familiarity and experience, yet these are the aspects of relativity that are popularized.
What you do not see in Sci-Fi is the intrinsic connection between relativistic
ideas and electromagnetism.  The same people who may say they do not believe in relativity would not say that they do not believe in their cell phone,
DVD player, personal computer, or television, yet the physics and engineering behind all of these and relativity are the same!

My personal justification for the Minkowski metric comes from consideration of the wave
nature of electromagnetism \citep{gabook:PJLorentzWave}.  A bit of study of electromagnetism shows that Maxwell's equations
for electric and magnetic fields requires that

\begin{equation}\label{eqn:schwartzchildMetric:120}
\begin{aligned}
\left(-\spacegrad^2 + \inv{c^2}\PDsQ{t}{}\right) \BE &= 0 \\
\left(-\spacegrad^2 + \inv{c^2}\PDsQ{t}{}\right) \BB &= 0
\end{aligned}
\end{equation}

Electromagnetic radiation (light) appears to be a fundamentally wavelike phenomena.  That is an idea that I think we are all fairly
comfortable with.  Mathematically, the mechanical introduction of an arbitrary change of variables \(x^{\mu} \rightarrow {x'}^{\mu}\)
should not change that.
If one describes all the types of change of variables such that
the wave equation retains its form in the second coordinate system, then application of the chain rule for the coordinate transformation
essentially imposes a Lorentz transformation constraint on this transformation.

The approach of using the wave equation as a motivator requires some calculus, and thus still is not something that
is good for a Layman's justification of the
Lorentz transformation and the corresponding Minkowski metric.  Einstein himself actually has a nice Layman's treatment of
special relativity in his book \citep{einstein2005rsa}.  The appendix of this book also has a simple derivation of the Lorentz transformation
in two variables that is well worth reading.
%  In that derivation it appeared to be obvious to Einstein that this transformation had to be linear,
%and , and
%my reading of this

\subsection{Minkowski metric justification by taking vector square roots of DeLambertian}

Also related to the wave equation one can justify the Minkowski metric by factoring the scalar wave equation (DeLambertian) operator

\begin{equation}\label{eqn:schwartzchildMetric:140}
\begin{aligned}
-\spacegrad^2 + \inv{c^2}\PDsQ{t}{} = -\sum_i \PDsQ{x^{i}}{} +\inv{c^2}\PDsQ{t}{}
\end{aligned}
\end{equation}

into a vector product.  If one writes the spacetime vector (with \(x^0 = ct\)) as

\begin{equation}\label{eqn:schwartzchildMetric:160}
\begin{aligned}
x = \gamma_{\mu} x^{\mu}
\end{aligned}
\end{equation}

A spacetime gradient operator can be defined that squares to \(\pm 1\) times the wave equation operator

\begin{equation}\label{eqn:schwartzchildMetric:180}
\begin{aligned}
\grad &= \gamma^{\mu} \PD{x^{\mu}}{} \\
\end{aligned}
\end{equation}

For the square of this operator to equal the wave equation operator we need two conditions

\begin{equation}\label{eqn:schwartzchildMetric:200}
\begin{aligned}
\gamma^{\mu} \cdot \gamma^{\nu} &= \delta^{\mu\nu} {\left(\gamma^{\mu}\right)}^2 \\
(\gamma^0)^2 (\gamma^i)^2 &= -1.
\end{aligned}
\end{equation}

With these we can then write the wave equation operator as

\begin{equation}\label{eqn:schwartzchildMetric:220}
\begin{aligned}
-\spacegrad^2 + \inv{c^2}\PDsQ{t}{} = \inv{(\gamma^0)^2} \grad^2
\end{aligned}
\end{equation}

There are an infinite number of ways to factor this scalar equation into a vector product, depending on the spacetime
coordinate systems used, but all of them vary by a Lorentz transformation!

Requiring that an orthonormal spacetime basis has a Minkowski metric signature \((\gamma^0)^2 (\gamma^i)^2 = -1\) has value
independent of any mechanics and relativistic consideration.  If nothing else the fact that this can be utilized to
summarize Maxwell's equations as the Clifford algebra product \citep{doran2003gap}

\begin{equation}\label{eqn:schwartzchildMetric:240}
\begin{aligned}
\grad F &= J/c \epsilon_0 \\
\end{aligned}
\end{equation}

Here \(\grad\) defined as above, and we have a bivector for the field, a vector for the current and charge density, and the
four space pseudoscalar ties the electric and magnetic field components together into a single complex number like quantity:

\begin{equation}\label{eqn:schwartzchildMetric:260}
\begin{aligned}
F &= \BE + I c \BB \\
J &= c \rho \gamma_0 + J^i \gamma_i \\
I &= \gamma_0 \wedge \gamma_1 \wedge \gamma_2 \wedge \gamma_3 \\
\end{aligned}
\end{equation}

The spatial vectors we are used to calculating with are expressed as spacetime bivectors
\begin{equation}\label{eqn:schwartzchildMetric:280}
\begin{aligned}
\sigma_i &= \gamma_i \wedge \gamma_0 \\
\sigma_i \cdot \sigma_j &= \delta_{ij} \\
\BE &= E^i \sigma_i \\
\BB &= B^i \sigma_i \\
\BJ &= J^i \sigma_i,
\end{aligned}
\end{equation}

where the bivector basis \(\{\sigma_i\}\) for the spacetime split behave in all respects like Euclidean vectors.

If nothing else the Minkowski metric idea that is at the root of all of this has got visible value as it brings together all
of electromagnetism under a single umbrella.

\section{Utilizing the Minkowski metric}

Presuming that some acceptable justification or motivation for the Minkowski metric has been accepted (I have outlined mine in the two sections above),
this allows for expression of four-vector distances when an orthonormal spacetime basis is used.

\subsection{Proper time and event arc length in SR}

The simplest example is for uniform motion along \(\gamma_1\), relative to a fixed point \(x^1\).  Here the event path for a particle can be written

\begin{equation}\label{eqn:schwartzchildMetric:300}
\begin{aligned}
x = c t \gamma_0 + (x^1 - v t) \gamma_1
\end{aligned}
\end{equation}

The differential distance for some parametrization of time \(t = t(\lambda)\) is thus

\begin{equation}\label{eqn:schwartzchildMetric:320}
\begin{aligned}
\frac{dx}{d\lambda} &= \frac{dt}{d\lambda} \left( c \gamma_0 - v \gamma_1 \right) \\
\left({\frac{dx}{d\lambda}}\right)^2 = {\frac{dx}{d\lambda}} \cdot {\frac{dx}{d\lambda}}
&= \left(\frac{dt}{d\lambda}\right)^2 \left( c^2 + v^2 (\gamma_1)^2 (\gamma_0)^2 \right) (\gamma_0)^2 \\
&= c^2 \left(\frac{dt}{d\lambda}\right)^2 \left( 1 - (v/c)^2 \right) (\gamma_0)^2 \\
\end{aligned}
\end{equation}

Writing in a signature independent fashion (since both \((\gamma_0)^2 = 1\), and \((\gamma_0)^2 = -1\) can be picked), the absolute
event arc length is

\begin{equation}\label{eqn:schwartzchildMetric:340}
\begin{aligned}
\left(\frac{ds}{d\lambda}\right)^2 &= \Abs{\frac{dx}{d\lambda}}^2  \\
&= (\gamma_0)^2 \left( {\frac{dx}{d\lambda}} \right)^2 \\
\delta s = c \int_{t_1}^{t_2} \sqrt{1 - (v/c)^2} dt
\end{aligned}
\end{equation}

Writing \(\delta s = c \delta\tau\), we have the proper time difference for fixed velocity motion

\begin{equation}\label{eqn:schwartzchildMetric:360}
\begin{aligned}
\delta \tau &= \sqrt{ 1 - (v/c)^2 } \delta t \\
\end{aligned}
\end{equation}

More generally, still employing the orthonormal Minkowski basis of SR for a particular event

\begin{equation}\label{eqn:schwartzchildMetric:380}
\begin{aligned}
x &= x^{\mu} \gamma_{\mu} \\
x^{\mu} &= x^{\mu}(\lambda)
\end{aligned}
\end{equation}

Taking derivatives along the path, we have

\begin{equation}\label{eqn:schwartzchildMetric:400}
\begin{aligned}
\frac{dx}{d\lambda} &= \frac{dx^{\mu}}{d\lambda} \gamma_{\mu} \\
\left(\frac{dx}{d\lambda}\right)^2 = \frac{dx}{d\lambda} \cdot \frac{dx}{d\lambda}
&= \gamma_{\mu} \cdot \gamma_{\nu} \frac{dx^{\mu}}{d\lambda} \frac{dx^{\nu}}{d\lambda} \\
&= (\gamma_{\mu})^2 \left( \frac{dx^{\mu}}{d\lambda} \right)^2.
\end{aligned}
\end{equation}

This yields the arc length along an arbitrary space time path

\begin{equation*}
\delta s = \int_{\lambda_1}^{\lambda_2} \sqrt{ (\gamma_{0})^2 (\gamma_{\mu})^2 \left( \frac{dx^{\mu}}{d\lambda} \right)^2} d\lambda.
\end{equation*}

As in the constant velocity case, the proper time is a velocity scaled event arc length, and for any path parametrization one has

\begin{equation}
\delta \tau = \inv{c} \int_{\lambda_1}^{\lambda_2} \sqrt{ (\gamma_{0})^2 (\gamma_{\mu})^2 \left( \frac{dx^{\mu}}{d\lambda} \right)^2} d\lambda.
\end{equation}

How is it reasonable to call this proper time, when it is really just represents 'four-vector-arc-length'?  The rational for this is that when the particle is at rest (\(dx^i/dt = 0\)) the proper time then becomes

\begin{equation}\label{eqn:schwartzchildMetric:420}
\begin{aligned}
\delta \tau
&= \inv{c} \int_{\lambda_1}^{\lambda_2} \sqrt{ (\gamma_{0})^4 \left(\frac{dx^0}{dt}\right)^2} dt \\
&= \inv{c} \int_{\lambda_1}^{\lambda_2} c \frac{dt}{dt} dt \\
&= \delta t
\end{aligned}
\end{equation}

The use of an orthonormal basis makes the metric tensor very simple.  It has a diagonal form and is almost identity

\begin{equation}\label{eqn:schwartzchildMetric:440}
\begin{aligned}
g_{\mu\nu} &= \gamma_{\mu} \cdot \gamma_{\nu} \\
&= \gamma_{\mu}^2 \delta_{\mu\nu} \\
&= (\gamma_0)^2
\begin{bmatrix}
1 & 0 & 0 & 0 \\
0 & -1 & 0 & 0 \\
0 & 0 & -1 & 0 \\
0 & 0 & 0 & -1 \\
\end{bmatrix}
\end{aligned}
\end{equation}

Proper time in terms of this metric tensor is thus

\begin{equation}
\delta \tau = \inv{c} \int_{\lambda_1}^{\lambda_2} \sqrt{ (\gamma_{0})^2 g_{\mu\nu} \frac{dx^{\mu}}{d\lambda} \frac{dx^{\nu}}{d\lambda} } d\lambda.
\end{equation}

\subsection{A more general metric tensor in SR}

A more interesting metric tensor than the almost diagonal form associated with our orthonormal frame can be had by
employing a more general basis \({f^{\mu}}\).  Write

\begin{equation}\label{eqn:schwartzchildMetric:460}
\begin{aligned}
x = f^{\mu} e_\mu = x^{\mu} \gamma_\mu
\end{aligned}
\end{equation}

The only restriction on the set of vectors \({e_\mu}\) is that they span the four-space.  That is sufficient to guarantee that a
reciprocal frame can be calculated at any point

\begin{equation}\label{eqn:schwartzchildMetric:480}
\begin{aligned}
e^\nu \cdot e_\mu = {\delta^\nu}_\mu.
\end{aligned}
\end{equation}

As in Euclidean space the reciprocal frame can then be used to calculate the coordinates of any given point

\begin{equation}\label{eqn:schwartzchildMetric:500}
\begin{aligned}
f^{\mu} = x \cdot e^\mu.
\end{aligned}
\end{equation}

Translation between this frame and the principle basis takes the form of a linear transformation

\begin{equation*}
e_\mu = a^{\mu\alpha} \gamma_\alpha
\end{equation*}

Given both sets of vectors this change of basis function \(a\) can be calculated by taking dot products

\begin{equation}\label{eqn:schwartzchildMetric:520}
\begin{aligned}
e_\mu \cdot \gamma^{\nu}
&= a^{\mu\alpha} \gamma_\alpha \cdot \gamma^{\nu} \\
&= a^{\mu\alpha} {\delta_\alpha}^\nu \\
&= a^{\mu\nu} \\
\end{aligned}
\end{equation}

\begin{equation}\label{eqn:schwartzchildMetric:540}
\begin{aligned}
e_\mu
&= \left(e_\mu \cdot \gamma^{\nu}\right) \gamma_\nu \\
&= \left(e_\mu \cdot \gamma_{\nu}\right) \gamma^\nu \\
\end{aligned}
\end{equation}

This is enough to calculate difference in position (provided \(f^{\mu}\) is not
a function of position).

\begin{equation}\label{eqn:schwartzchildMetric:560}
\begin{aligned}
\left(\frac{dx}{d\lambda} \right)^2
&=
e_\mu \cdot e_\nu
\frac{df^{\mu}}{d\lambda} \frac{df^{\nu}}{d\lambda} \\
&=
\left(e_\mu \cdot \gamma^{\alpha}\right)
\left(e_\nu \cdot \gamma_{\beta}\right)
\gamma_\alpha \cdot \gamma^\beta
\frac{df^{\mu}}{d\lambda} \frac{df^{\nu}}{d\lambda} \\
&=
\left(e_\mu \cdot \gamma^{\alpha}\right)
\left(e_\nu \cdot \gamma_{\alpha}\right)
\frac{df^{\mu}}{d\lambda} \frac{df^{\nu}}{d\lambda} \\
\end{aligned}
\end{equation}

This gives us the proper time with respect to these more general coordinates and their associated basis vectors

\begin{equation}
\delta \tau = \inv{c} \int_{\lambda_1}^{\lambda_2}
\sqrt{
(\gamma_{0})^2
\left(e_\mu \cdot \gamma^{\alpha}\right)
\left(e_\nu \cdot \gamma_{\alpha}\right)
\frac{df^{\mu}}{d\lambda} \frac{df^{\nu}}{d\lambda} } d\lambda.
\end{equation}

For this mess of dot products, introduction of a tensor

\begin{equation}
g_{\mu\nu} =
(\gamma_{0})^2
\left(e_\mu \cdot \gamma^{\alpha}\right)
\left(e_\nu \cdot \gamma_{\alpha}\right)
\end{equation}

allows for the expressing the proper time for these
general coordinates in a more compact form

\begin{equation}\label{eqn:schwartzchild_metric:SRmetric}
\delta \tau = \inv{c} \int_{\lambda_1}^{\lambda_2}
\sqrt{
g_{\mu\nu}
\frac{df^{\mu}}{d\lambda} \frac{df^{\nu}}{d\lambda} } d\lambda.
\end{equation}

Both of the Rindler and the Schwartzchild metrics as stated below have
the general form of this structure of equation
\eqnref{eqn:schwartzchild_metric:SRmetric} (this is in fact a more general form since both those metrics are diagonal).

\subsection{Position variation of SR frame vectors}

This is all assuming that the spacetime frame is not a function of position or time, or the variation of the frame vectors
is so small that it can be neglected.

In the more general case when the derivatives of the frame vectors
are significant those will have to be considered to calculate distance.

Lets see what the SR metric looks like in this case.

Again write

\begin{equation}\label{eqn:schwartzchildMetric:580}
\begin{aligned}
x = f^\mu e_\mu
\end{aligned}
\end{equation}

Allowing both \(f^\mu = f^\mu(\lambda)\) and \(e_\mu = e_\mu(f^\nu)\) to vary with position a differential change in position
with respect to parameter \(\lambda\) is thus

\begin{equation}\label{eqn:schwartzchildMetric:600}
\begin{aligned}
\frac{dx}{d\lambda}
&= \frac{d f^\mu}{d\lambda} e_\mu + f^\nu \PD{f^\mu}{e_\nu} \frac{d f^\mu}{d\lambda} \\
&= \frac{d f^\mu}{d\lambda} \left( e_\mu + f^\nu \PD{f^\mu}{e_\nu} \right)
\end{aligned}
\end{equation}

Introducing shorthand \(\partial_\mu \equiv \PD{f^{\mu}}{}\), and \(\fdot^{\mu} = \frac{df^\mu}{d\lambda}\), the squared magnitude is

\begin{equation}\label{eqn:schwartzchildMetric:620}
\begin{aligned}
\left(\frac{dx}{d\lambda} \right)^2 &=
\left( e_\mu + f^\alpha \partial_\mu {e_\alpha} \right) \cdot \left( e_\nu + f^\beta \partial_\nu {e_\beta} \right)
\fdot^\mu \fdot^\nu \\
&=
\left(
  e_\mu \cdot e_\nu
+ f^\beta e_\mu \cdot (\partial_\nu {e_\beta})
+ f^\alpha (\partial_\mu {e_\alpha}) \cdot e_\nu
+ f^\alpha f^\beta (\partial_\mu {e_\alpha}) \cdot (\partial_\nu {e_\beta})
\right)
\fdot^\mu \fdot^\nu \\
\end{aligned}
\end{equation}

The dot products term here is symmetric, so one can write

\begin{equation}\label{eqn:schwartzchildMetric:640}
\begin{aligned}
g_{\mu\nu}(f^\sigma) =
  e_\mu \cdot e_\nu
+ 2 f^\beta e_\mu \cdot (\partial_\nu {e_\beta})
%+ f^\alpha (\partial_\mu {e_\alpha}) \cdot e_\nu
+ f^\alpha f^\beta (\partial_\mu {e_\alpha}) \cdot (\partial_\nu {e_\beta})
\end{aligned}
\end{equation}

and express the incremental arc length (ignoring potential metric signature sign adjustment) along the curve as

\begin{equation}\label{eqn:schwartzchildMetric:660}
\begin{aligned}
\left(\frac{dx}{d\lambda} \right)^2 (d\lambda)^2 &= g_{\mu\nu}(f^\sigma) \fdot^\mu \fdot^\nu
\end{aligned}
\end{equation}

This is the completely general kinetic form for the Lagrangian of \eqnref{eqn:schwartzchild_metric:keLagrangian}.  The variation results
for this equation that are summarized in \eqnref{eqn:schwartzchild_metric:geodesics} are therefore also equations of motion for special relativity
when we allow for expressing the four vector in generalized coordinates.  They happen to also be how the equations of motion
of general relativity are also expressed.  This shows that there is likely a geometric special relativistic interpretation
of any general relativity metric.  The Rindler metric calculation below will illustrate this well due to its simplicity.

\section{Some GR calculations}

\subsection{Schwartzchild Metric}

After too much thought about the geometrical origins of more general metrics in
relativity, lets take one of them and run with it,
working through the Euler-Lagrange equations

\begin{equation}\label{eqn:schwartzchildMetric:680}
\begin{aligned}
\PD{q}{\LL} = \frac{d}{d\tau}\left(\PD{\qdot}{\LL}\right)
\end{aligned}
\end{equation}

for what Lut calls the Schwartzchild metric.

\begin{equation}\label{eqn:schwartzchildMetric:700}
\begin{aligned}
-\CC (d\tau)^2 &= -\CC a(r) (dt)^2 + {b(r)} (dr)^2 + r^2(d\theta)^2 \\
a(r) &= 1 - \kappa/r \\
b(r) &= \inv{a(r)} = \frac{r}{r-k}
\end{aligned}
\end{equation}

Form the Lagrangian

\begin{equation}\label{eqn:schwartzchild_metric:sLagrangian}
\LL = -\CC = -\CC a \tdot^2 + b \rdot^2 + r^2 \dottheta^2
\end{equation}

Some intermediate calculations will be useful

\begin{equation}\label{eqn:schwartzchildMetric:720}
\begin{aligned}
\PD{r}{a} &= \kappa/r^2 \\
\PD{r}{b} &= -\kappa/(r-k)^2 \\
\adot &= \frac{\kappa \rdot}{r^2} \\
\bdot &= \frac{-\kappa \rdot}{(r-k)^2}
\end{aligned}
\end{equation}

First calculate the EOMs for the cyclic coordinates

\begin{equation}\label{eqn:schwartzchildMetric:740}
\begin{aligned}
\PD{\theta}{\LL} &= \frac{d}{d\tau}\left(\PD{\dottheta}{\LL}\right) \\
0 &= (2 r^2 \dottheta)'
\end{aligned}
\end{equation}
\begin{equation}\label{eqn:schwartzchildMetric:760}
\begin{aligned}
\PD{t}{\LL} &= \frac{d}{d\tau}\left(\PD{\tdot}{\LL}\right) \\
0 &= (- 2 c^2 a \tdot)'
\end{aligned}
\end{equation}

Introducing two arbitrary integration constants we have

\begin{equation}\label{eqn:schwartzchild_metric:cyclic}
\begin{aligned}
\dottheta &= \frac{A}{r^2} \\
\tdot &= \frac{ T }{a}
\end{aligned}
\end{equation}

The equations for the non-cyclic coordinate \(r\) is

\begin{equation}\label{eqn:schwartzchildMetric:780}
\begin{aligned}
\PD{r}{\LL} &= \frac{d}{d\tau}\left(\PD{\rdot}{\LL}\right) \\
-c^2 \PD{r}{a} \tdot^2 + \PD{r}{b} \rdot^2 + 2 r \dottheta^2 &= \left( 2 b \rdot \right)' = 2 \bdot \rdot + 2 b \rddot \\
-c^2 \frac{\kappa}{r^2} \tdot^2 - \frac{\kappa}{(r-k)^2} \rdot^2 + 2 r \dottheta^2 &= -2 \frac{\kappa \rdot^2}{(r-k)^2} + 2 b \rddot \\
-\inv{2} c^2 \frac{a \kappa}{r^2} \tdot^2 - \inv{2} \frac{a \kappa}{(r-k)^2} \rdot^2 + a r \dottheta^2 +\frac{a \kappa \rdot^2}{(r-k)^2} &= \rddot \\
-\inv{2} c^2 \frac{\kappa T^2}{a r^2} + \inv{2} \frac{a \kappa}{(r-k)^2} \rdot^2 + a \frac{A^2}{r^3} &= \\
\end{aligned}
\end{equation}

From the Lagrangian itself \eqnref{eqn:schwartzchild_metric:sLagrangian}, and the
integrated cyclic equations \eqnref{eqn:schwartzchild_metric:cyclic} one can eliminate
the \(\rdot^2\) term above

\begin{equation}\label{eqn:schwartzchildMetric:800}
\begin{aligned}
b \rdot^2 &= -\CC +\CC a \tdot^2 - r^2 \dottheta^2  \\
\rdot^2
&= - a \CC +\CC a^2 \tdot^2 - a r^2 \dottheta^2 \\
&= \CC(T^2 - a) - a \frac{A^2}{r^2} \\
\end{aligned}
\end{equation}

\begin{equation}\label{eqn:schwartzchildMetric:820}
\begin{aligned}
\rddot
&= -\inv{2} c^2 \frac{\kappa T^2}{a r^2} + \inv{2} \frac{a \kappa}{(r-\kappa)^2} \left( \CC(T^2 - a) - a \frac{A^2}{r^2} \right) + a \frac{A^2}{r^3} \\
&=
\mathLabelBox{
- \inv{2} \CC \frac{\kappa T^2}{(r-\kappa) r}
+ \inv{2} \CC \frac{\kappa T^2}{(r-\kappa) r}
}{\(=0\)}
- \inv{2} \frac{\kappa c^2}{r^2}
- \inv{2} \frac{\kappa A^2}{r^4}
+ \frac{(r-\kappa)A^2}{r^4} \\
\end{aligned}
\end{equation}

A final collection of terms yields

\begin{equation}\label{eqn:schwartzchildMetric:840}
\begin{aligned}
\rddot &=
- \inv{2} \frac{\kappa c^2}{r^2}
+ \frac{A^2}{r^3}
- \frac{3 \kappa A^2}{2 r^4}
\end{aligned}
\end{equation}

Now, the interesting thing here is that the metric itself can be considered a source of gravitational acceleration.  Let \(\kappa c^2/2 = G M\), and form the proper acceleration

\begin{equation}\label{eqn:schwartzchildMetric:860}
\begin{aligned}
\rddot &=
- \frac{G M}{r^2}
+ \frac{A^2}{r^3}
- \frac{3 G M A^2}{c^2 r^4}
\end{aligned}
\end{equation}

Note the similarity now to Newtonian gravity.

Lets also eliminate the arbitrary integration constant, using the angular velocity at a specific reference point \(A = \dottheta r^2 = \Omega_0 (r_0)^2\)

\begin{equation}\label{eqn:schwartzchild_metric:schacc}
\begin{aligned}
 \rddot &=
- \frac{G  M }{r^2}
+ \frac{ (\Omega_0)^2 (r_0)^4}{r^3}
- \inv{c^2} 3 G M  ({\Omega}_0)^2 \left( \frac{r_0}{r} \right)^4
\end{aligned}
\end{equation}

This calculation is missing physics content.  A good comparison to radial EOM in Newtonian physics ought to be done to see
which of these terms can also be found in a non-relativistic treatment.  Lut also suggests that \eqnref{eqn:schwartzchild_metric:schacc} can
probably be applied to calculate planar precession, which would be a cool application.

Blatantly missing here is an understanding of how the metric relates to the mass distribution (ie: this is apparently for
a spherical mass distribution).  A treatment that seems quite readable is the following translation of Schwartzchild's original
paper \citep{schwarzschild-1916-1916}.

Question: can the Schwartzchild metric be formulated in terms of a position dependent variable basis as the Rindler metric can?

\subsection{Rindler Metric}

The Rindler metric has only \(g_{00} = g_{00}(x^1)\) different from unity.

Corresponding to a generalized basis

\begin{equation}\label{eqn:schwartzchildMetric:880}
\begin{aligned}
e_0 &= \sqrt{g_{00}} \gamma_0 \\
e_i &= \gamma_i
\end{aligned}
\end{equation}

Without specifying this function specifically, lets try the calculation.
Writing \(f(x) = g_{00}(x^1)\), the arc length and corresponding Lagrangian is

\begin{equation}\label{eqn:schwartzchildMetric:900}
\begin{aligned}
-\CC (d\tau)^2 &= -\CC (dt^0)^2 f(x) + dx^2 \\
\LL &= -\CC = -\CC \tdot^2 f(x) + \xdot^2
\end{aligned}
\end{equation}

The generalized time coordinate is cyclic:
\begin{equation}\label{eqn:schwartzchildMetric:920}
\begin{aligned}
\left( - 2f(x) \CC \tdot \right)' &= 0 \\
- 2f(x) \CC \tdot &= -2 \CC \kappa \\
\tdot &= \frac{\kappa}{f(x)}.
\end{aligned}
\end{equation}

For the \(x\) coordinate we have
\begin{equation}\label{eqn:schwartzchildMetric:940}
\begin{aligned}
2 \xddot &= - f'(x) \CC \tdot^2 \\
\xddot
&= - \inv{2} f'(x) \CC \tdot^2 \\
&= - \frac{f'(x) \CC \kappa^2}{f^2} \\
\end{aligned}
\end{equation}

Which is
\begin{equation}\label{eqn:schwartzchildMetric:960}
\begin{aligned}
\frac{d^2 x}{d\tau^2} &= \CC \kappa^2 \frac{d}{dx}\left(\inv{2 f(x)}\right).
\end{aligned}
\end{equation}

For \(f(x) = a + bx\), this is

\begin{equation}\label{eqn:schwartzchildMetric:980}
\begin{aligned}
\frac{d^2 x}{d\tau^2}
&= - \CC \kappa^2 \frac{b}{2 (a+bx)^2}  \\
\end{aligned}
\end{equation}

For \(f(x) = (a + bx)^2\), what I believe Lut used, this is

\begin{equation}\label{eqn:schwartzchildMetric:1000}
\begin{aligned}
\frac{d^2 x}{d\tau^2}
&= - \CC \kappa^2 \frac{b}{ (a+bx)^3} \\
\end{aligned}
\end{equation}

For \(f(x) = (a + bx)^{1/2}\), this is

\begin{equation}\label{eqn:schwartzchildMetric:1020}
\begin{aligned}
\frac{d^2 x}{d\tau^2}
&= \inv{4} \CC \kappa^2 b (a + bx)^{-3/2}
\end{aligned}
\end{equation}

FIXME: I can not reproduce the \(\xddot \propto 1/\sqrt{f}\) result that Lut had, nor what I initially calculated on paper.  My initial paper calculation looked wrong later when typing up, since I had messed up polynomial powers (probably since
I was expecting a specific answer).
The above was recalculated without specifying f(x) upfront to make it
harder to mess up the powers.

Think that I saw in the wiki page on the Rindler metric that there were both
\(g_{00}\) and \(g_{11}\) terms (with square roots) in both.  Try with that, and
also do the calculations to see that those match the \(\Gamma\) gravitational
field equations as specified in \citep{schwarzschild-1916-1916}.

As far as interpretation goes...
I think this can be interpreted in the geometrical fashion that intuition
was telling me was there.
From a strictly SR point of view, if you calculate the equations of motion for a particle in a frame where the time basis vector increases with position, a particle at rest in that frame is observed to be accelerating from an external (w/ constant basis vectors) frame.  This sort of general coordinate system variation
can not necessarily be interpreted as an Einstein gravitational field
since he has constraints on the allowed metrics.  Working through some
examples of that field calculation with various metrics should be helpful
to get a feel for things.

FIXME: firm up this interpretive statement with the math to make it more meaningful, and
consider physical examples of motion in the corresponding systems.

%\bibliographystyle{plain}
%\bibliographystyle{plainnat} % supposed to allow for \url use.
%\bibliography{myrefs}      % expects file "myrefs.bib"

%\end{document}               % End of document.
