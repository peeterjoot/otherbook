%
% Copyright � 2012 Peeter Joot.  All Rights Reserved.
% Licenced as described in the file LICENSE under the root directory of this GIT repository.
%

%
%
%\documentclass{article}

%\input{../peeters_macros.tex}
%\input{../peeters_macros2.tex}

%\usepackage[bookmarks=true]{hyperref}

%\usepackage{color,cite,graphicx}
   % use colour in the document, put your citations as [1-4]
   % rather than [1,2,3,4] (it looks nicer, and the extended LaTeX2e
   % graphics package.
%\usepackage{latexsym,amssymb,epsf} % do not remember if these are
   % needed, but their inclusion can not do any damage


%\chapter{Time for two bodies to converge in space}
\label{chap:twobodies}
%\author{Peeter Joot \quad peeterjoot@protonmail.com}
\date{ Dec 15, 2008.  \(RCSfile: twobodies.tex,v \) Last \(Revision: 1.10 \) \(Date: 2009/06/14 23:51:45 \) }

%\begin{document}

%\maketitle{}

%\tableofcontents

\section{Motivation}

An attempt to calculate the time for masses to converge in space under their own gravity.  Did this initially without actually evaluating the final
integral or putting in numbers.  Once I did that I saw that the approach is totally busted.  Archive this for later to perhaps revisit if I feel
inclined.

\section{The garbage}

Assuming that the initial relative velocities of the two bodies lies along
the line between them this can be treated as a one dimensional problem.

We have two force equations

\begin{equation}\label{eqn:twobodies:20}
\begin{aligned}
\text{Force on} \quad m_2 &= - G m_1 m_2 (x_2 - x_1)^{-2} = m_2 {x_2}'' \\
\text{Force on} \quad m_1 &=  G m_1 m_2 (x_2 - x_1)^{-2} = m_1 {x_1}''
\end{aligned}
\end{equation}

Subtraction (scaled) of these two equations leaves an integration problem for a difference of position variable \(u = x_2 - x_1\)

\begin{equation}\label{eqn:twobodies:40}
\begin{aligned}
- G (x_2 - x_1)^{-2} (m_1 + m_2)= (x_2 - x_1)'' \\
\implies
- G (m_1 + m_2) \inv{u^2} = u'' \\
\end{aligned}
\end{equation}

The chain rule can be used for a first elimination of a time derivative

\begin{equation}\label{eqn:twobodies:60}
\begin{aligned}
\frac{d}{dt} = \frac{du}{dt} \frac{d}{du}
\end{aligned}
\end{equation}

With \(k = G(m_1 + m_2)\), this is

\begin{equation}\label{eqn:twobodies:80}
\begin{aligned}
\frac{du}{dt} \frac{d}{du} \left(\frac{du}{dt}\right) &= -k u^{-2}
\end{aligned}
\end{equation}

Introducing a difference of position "velocity" the LHS of this is

\begin{equation}\label{eqn:twobodies:100}
\begin{aligned}
\frac{du}{dt} \frac{d}{du} \left(\frac{du}{dt}\right)
&= v \frac{dv}{du} \\
&= \frac{d}{du} \left( \inv{2}v^2 \right) \\
\end{aligned}
\end{equation}

So now we have something that can be directly integrated:

\begin{equation}\label{eqn:twobodies:120}
\begin{aligned}
\inv{2}v^2 &= k \inv{u} + \inv{2}C
\end{aligned}
\end{equation}

Fixing the constant in terms of the initial position and speed this is

\begin{equation}\label{eqn:twobodies:velocity}
\begin{aligned}
\frac{du}{dt} &= \pm \sqrt{v_0^2 + 2 G M \left(\inv{u} - \inv{u_0}\right)}
\end{aligned}
\end{equation}

and we can integrate for the time

\begin{equation}\label{eqn:twobodies:140}
\begin{aligned}
\delta t
&= \pm \int_{u_0}^{u_1} du \left(v_0^2 + 2 G M \left(\inv{u} - \inv{u_0}\right)\right)^{-1/2} \\
\end{aligned}
\end{equation}

Perhaps more interesting than evaluating this integral or getting some numbers is to think through the physical meaning for the \(\pm\).  It makes sense in the velocity \eqnref{eqn:twobodies:velocity} since the particles can be initially moving towards or away from each other (or stationary).  Intuition says that negative time must mean that the particles already passed the desired threshold, and are continuing on towards each other.  For now, I will just ignore the negative case.  Writing

\begin{equation}\label{eqn:twobodies:160}
\begin{aligned}
\alpha = {v_0}^2 - \frac{2GM}{u_0}
\end{aligned}
\end{equation}

This is
\begin{equation}\label{eqn:twobodies:180}
\begin{aligned}
\delta t
&=  \int_{u_0}^{u_1} du \left(\alpha + \frac{2 G M}{u} \right)^{-1/2} \\
&=  \int_{u_0}^{u_1} du \sqrt{\frac{u}{\alpha u + 2 G M}} \\
&=  \inv{\sqrt{\Abs{\alpha}}} \int_{u_0}^{u_1} \frac{du \sqrt{u}}{\sqrt{\sgn(\alpha) u + \Abs{2 G M/\alpha}}}
\end{aligned}
\end{equation}

Now write \(\beta = \Abs{2Gm/\alpha}\), and substitute \(u = \beta x\) for
\begin{equation}\label{eqn:twobodies:200}
\begin{aligned}
\delta t
&=  \inv{\sqrt{\Abs{\alpha}}} \int_{x= u_0/\beta}^{u_1/\beta} \frac{\beta dx \sqrt{\beta x}}{\sqrt{\sgn(\alpha) \beta x + \beta}} \\
&=  \frac{\beta}{\sqrt{\Abs{\alpha}}} \int_{x= u_0/\beta}^{u_1/\beta} \frac{dx \sqrt{ x}}{\sqrt{\sgn(\alpha)  x + 1}} \\
\end{aligned}
\end{equation}

This is now in a good form to check a table of integrals, and after a bit of manipulation we have

\begin{equation}\label{eqn:twobodies:220}
\begin{aligned}
\begin{array}{l l}
\int \frac{dx \sqrt{ x}}{\sqrt{1 - x}} = - \sqrt{x(1-x)} + \tan^{-1}{\sqrt{\frac{1-x}{x}}} & \quad \mbox{if \(\alpha < 0\)} \\
\int \frac{dx \sqrt{ x}}{\sqrt{1 - x}} = \sqrt{x(1-x)} + \tanh^{-1}{\sqrt{\frac{1+x}{x}}} & \quad \mbox{if \(\alpha > 0\)} \\
\end{array}
\end{aligned}
\end{equation}

The \(\alpha < 0\) case (ie: \(v_0 = 0\)) is probably the more interesting.

Punching in some numbers.  Two 1kg masses, one meter apart with zero initial velocity, and the time to get to 20 cm separation (ie: time for basketball sized objects to touch).  This gives 16.1 hours.  Also looks like with \(v_0 = 0\) 1 meter is a special distance, and evaluation of the integral at that distance yields zero.  For \(x>1\) looks like we need a different variation of the integral (to avoid imaginary numbers in the square roots).  Now, curiously if I set the initial distance to \(.5 m\), the time with this integral goes up (54 hrs)?  I must have it wrong.

Let us cut some complexities out by taking \(v_0 = 0\) explicitly.  This then gives us

\begin{equation}\label{eqn:twobodies:240}
\begin{aligned}
\frac{du}{dt}
&= \sqrt{2GM} \sqrt{\inv{u} - \inv{u_0}} \\
&= \sqrt{2GM} \sqrt{\frac{u_0 - u}{u u_0}} \\
&= \sqrt{\frac{2GM}{u_0}} \sqrt{\frac{1 - u/u_0}{u}} \\
\end{aligned}
\end{equation}

Ah, this shows why this does not work.  The numerator here is zero for \(u = u_0\), so we can not invert to integrate for time.  The whole approach is busted.

%\bibliographystyle{plainnat}
%\bibliography{myrefs}

%\end{document}
