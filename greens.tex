%
% Copyright � 2012 Peeter Joot.  All Rights Reserved.
% Licenced as described in the file LICENSE under the root directory of this GIT repository.
%

%
%
%\documentclass{article}

%\input{../peeters_macros.tex}
%\input{../peeters_macros2.tex}

%\usepackage[bookmarks=true]{hyperref}

%\usepackage{color,cite,graphicx}
   % use colour in the document, put your citations as [1-4]
   % rather than [1,2,3,4] (it looks nicer, and the extended LaTeX2e
   % graphics package.
%\usepackage{latexsym,amssymb,epsf} % do not remember if these are
   % needed, but their inclusion can not do any damage


\chapter{Green's functions}
\label{chap:greens}
%\author{Peeter Joot \quad peeterjoot@protonmail.com}
\date{ Dec 19, 2008.  \(RCSfile: greens.tex,v \) Last \(Revision: 1.10 \) \(Date: 2009/08/20 02:24:45 \) }

%\begin{document}
%\maketitle{}
%\tableofcontents

\section{Motivation}

For electromagnetism the vector potential equation to solve is

\begin{equation}\label{eqn:greens:20}
\begin{aligned}
\grad (\grad \wedge A) = \frac{J}{c \epsilon_0}
\end{aligned}
\end{equation}

Imposing a gauge condition \(\grad \cdot A\) on the solutions \(A\), we have

\begin{equation}\label{eqn:greens:40}
\begin{aligned}
\grad^2 A = \frac{J}{c \epsilon_0}
\end{aligned}
\end{equation}

Which gives us our four scalar potential equations

\begin{equation}\label{eqn:greens:60}
\begin{aligned}
\left(\inv{c^2}\PDsq{t}{} - \sum_k \PDsq{x^k}{} \right) A^0 &= \frac{\rho}{\epsilon_0} \\
\left(\inv{c^2}\PDsq{t}{} - \sum_k \PDsq{x^k}{} \right) A^i &= \frac{J^i}{c \epsilon_0} \\
\end{aligned}
\end{equation}

One form of solution, found for example in \citep{feynman1963flp} for these equations is the retarded time potentials.  For example, with \(\phi = A^0\), the retarded solution for \(\phi\) at some time \(t\), is expressed as an integral over all space

\begin{equation}\label{eqn:greens:80}
\begin{aligned}
\phi(\Bx, t) &= \inv{4 \pi \epsilon_0} \int \frac{\rho(\Bx', t - \Abs{\Bx-\Bx'}/c)}{\Abs{\Bx - \Bx'}} dV'
\end{aligned}
\end{equation}

The first time I saw this equation, I was struck by the relativistic nature of the solution.  Only the charges with speed of light separation from the point of interest has an effect on the field at that point.

Unfortunately understanding the origin of those solutions, as covered in \citep{FitzRelEandM}, appears to require Green's functions, Fourier transforms and a whole mess of complex seeming stuff.

Intuition tells me that relativistic treatment of the electrostatics potential solution (ie: Lorentz boosting the Coulomb statics potential) can be used to develop the retarded time solutions, and that will be worth a try.  However, it also appears that direct study of the Green's function tools will be worthwhile.

These notes will examine topics related to Green's functions, with the eventual goal of building towards these retarded time potential solutions.

\section{Green's functions}

The basic idea is given a linear differential operator \(\LL\), and the Dirac delta function \(\delta\), a Green's function solution to the operator equation is

\begin{equation}\label{eqn:greens:100}
\begin{aligned}
\LL G(x,s) = \delta(x-s)
\end{aligned}
\end{equation}

Using convolution the impulse (delta) function can be used to form arbitrary driving functions, and the general solution, by superposition then only requires an equivalent convolution integration.

Examples of possible differential operators are

\begin{equation}\label{eqn:greens:120}
\begin{aligned}
\LL &= \frac{d}{dx} + a \\
\LL &= a \PDsq{x}{} + b \PDsq{y}{} \\
\LL &= \spacegrad^2 - \inv{v^2}\PDsq{t}{}\\
\LL &= \grad^2 \\
\end{aligned}
\end{equation}

The last of which is the desired operator for the electromagnetism case.

\section{Solving linear differential equations}

We do not necessarily need Green's functions and transform theory explicitly to solve linear equations.  Lets review some simple cases to see how to directly solve some of these.

\subsection{Simplest case}

The simplest case is a homogeneous linear first order equation of one variable.

\begin{equation}\label{eqn:greens:140}
\begin{aligned}
f' + a f = 0
\end{aligned}
\end{equation}

The usual exponential characteristic solution method solves this one

\begin{equation}\label{eqn:greens:160}
\begin{aligned}
f &= e^{rx} \\
\implies \\
(r + a) e^{rx} &= 0
\end{aligned}
\end{equation}

So a homogeneous solution is

\begin{equation}\label{eqn:greens:180}
\begin{aligned}
f = e^{-ax}
\end{aligned}
\end{equation}

The next level of complexity is the inhomogeneous case, with a constant forcing function

\begin{equation}\label{eqn:greens:200}
\begin{aligned}
f' + a f = b
\end{aligned}
\end{equation}

This is separable and the solution follows from direct integration

\begin{equation}\label{eqn:greens:220}
\begin{aligned}
\int \frac{df}{b - a f} &= \int dx \\
\inv{-a}\ln(b - af) &= x - \inv{a} ln(-B) \\
\end{aligned}
\end{equation}

So our complete solution is

\begin{equation}\label{eqn:greens:240}
\begin{aligned}
f &= A e^{-ax} + \inv{a}\left(b + B e^{-a x}\right)
\end{aligned}
\end{equation}

\subsubsection{With a sampling impulse function}

Now consider a unit area rectangular spike driving function of width \(\epsilon\), and height \(1/\epsilon\).  Will this be a reasonable approximation of a delta function when \(\epsilon\) is let approach zero?

\begin{equation}\label{eqn:greens:260}
\begin{aligned}
b(x) =
\left\{
\begin{array}{l l}
1/{\epsilon} & \quad \mbox{if \(x \in \left[-\epsilon/2, \epsilon/2 \right]\)} \\
0              & \quad \mbox{otherwise}
\end{array} \right.
\end{aligned}
\end{equation}

With a degree of freedom for the homogeneous solution in each of the three ranges, and one degree of freedom for the inhomogeneous part in the impulse interval we have the following general solution

\begin{equation}\label{eqn:greens:280}
\begin{aligned}
f(x) =
\left\{
\begin{array}{l l}
A_{-} e^{-a(x + \epsilon/2)} & \quad x < -\epsilon/2 \\
%\inv{a \epsilon}
B e^{-ax} + \inv{a}\left(\inv{\epsilon} + C e^{-a x}\right)& \quad x \in \left[-\epsilon/2, \epsilon/2 \right] \\
A_{+} e^{-a(x - \epsilon/2)} & \quad x > \epsilon/2 \\
\end{array} \right.
\end{aligned}
\end{equation}

%There is actually one degree of freedom too many in the middle term as can be seen by regrouping slightly
%\begin{align*}
%\inv{a \epsilon} B e^{-ax} + \inv{a}\left(\inv{\epsilon} + C e^{-a x}\right)
%&= \inv{a \epsilon} (B+C) e^{-ax} + \inv{a \epsilon} \\
%&= \inv{a \epsilon} \left( (B + C)e^{-ax} + 1 \right)
%\end{align*}

This is not necessarily a continuous function, and there is in fact a degree of freedom too many in the middle term above.  Suppose that one wanted this term
to take values \(B_{-}\), and \(B_{+}\) at the boundaries of the impulse interval, one is then left with the following set of simultaneous equations

\begin{equation}\label{eqn:greens:300}
\begin{aligned}
B_{-} &= \left(B + \inv{a \epsilon } C \right) e^{a \epsilon/2} + \inv{a \epsilon} \\
B_{+} &= \left(B + \inv{a \epsilon } C \right) e^{-a \epsilon/2} + \inv{a \epsilon} \\
\end{aligned}
\end{equation}

As a vector equation this is
\begin{equation}\label{eqn:greens:320}
\begin{aligned}
a \epsilon
\begin{bmatrix}
B_{-} \\
B_{+}
\end{bmatrix}
=
\left(a \epsilon B + C \right)
\begin{bmatrix}
e^{a \epsilon/2} \\
e^{-a \epsilon/2}
\end{bmatrix}
+
\begin{bmatrix}
1 \\
1
\end{bmatrix}
\end{aligned}
\end{equation}

and we can wedge with \((1,1)\) to eliminate it, leaving

\begin{equation}\label{eqn:greens:340}
\begin{aligned}
a \epsilon
(B_{-} - B_{+})
=
\left(a \epsilon B + C \right)
(e^{a \epsilon/2} - e^{-a \epsilon/2})
\end{aligned}
\end{equation}

So we have

\begin{equation}\label{eqn:greens:360}
\begin{aligned}
B + \inv{a \epsilon}C = \frac{B_{-} - B_{+}}{2 \sinh(a \epsilon/2)}
\end{aligned}
\end{equation}

and the solution to our equation is therefore

\begin{equation}\label{eqn:greens:380}
\begin{aligned}
f(x) =
\left\{
\begin{array}{l l}
A_{-} e^{-a(x + \epsilon/2)} & \quad x < -\epsilon/2 \\
\frac{B_{-} - B_{+}}{2 \sinh(a \epsilon/2)} e^{-ax} + \inv{a \epsilon} & \quad x \in \left[-\epsilon/2, \epsilon/2 \right] \\
A_{+} e^{-a(x - \epsilon/2)} & \quad x > \epsilon/2 \\
\end{array} \right.
\end{aligned}
\end{equation}

However, with the middle term being a function of only the difference between the endpoints, means that we must have an additional constraint on the endpoints.  In general one could make this match the a desired value at only one of the endpoints, and then the other is then determined.  How about picking the middle term so that its value at \(x=0\) is the midpoint between \(A_{-}\) and \(A_{+}\).  If one writes \(f = \mu e^{-ax} + 1/a \epsilon\) we have

\begin{equation}\label{eqn:greens:400}
\begin{aligned}
\mu + \inv{a\epsilon} &= \inv{2}\left(A_{-} + A_{+}\right) \\
\implies \\
f &= \inv{2}\left(A_{-} + A_{+}\right) e^{-ax} + \inv{a \epsilon} \left(1 - e^{-a x}\right)
\end{aligned}
\end{equation}

Now, this was for only the impulse interval.  If however, we have \(A = A_{-} = A_{+}\), then writing \(u(x)\) for the unit step function, we have for the complete interval the following general solution

\begin{equation}\label{eqn:greens:420}
\begin{aligned}
f &= A e^{-ax} + \inv{2 a \epsilon} \left(1 - e^{-a x}\right)\left( u(\epsilon/2 + x) + u(\epsilon/2 -x)\right)
\end{aligned}
\end{equation}

This finally has a desirable form, with the sum of a regular old homogeneous solution, plus a specific solution.  The specific solution here also has the role of the Green's function for our delta like impulse function.  Doing a convolution sum with this impulse function has a sampling like action, but it is not clear how to take the limit of the Green's like function as \(\epsilon \rightarrow 0\).

Seeing the delta like function here suggests that the proper Green's function for this operator may in fact be a weighted delta function.  Specifically if we had

\begin{equation}\label{eqn:greens:440}
\begin{aligned}
f' + a f &= \delta(x) \\
\end{aligned}
\end{equation}

Then is the corresponding general solution in terms of homogeneous solution and a Green's function like so:
\begin{equation}\label{eqn:greens:460}
\begin{aligned}
f &= A e^{-ax} + \inv{a} \left(1 - e^{-a x}\right) \delta(x) \\
\end{aligned}
\end{equation}

How would one even see if this makes sense (ie: how does one take the derivative of a delta function)?  Probably need the convolution before this would make sense.  Let us try this.

%\bibliographystyle{plainnat}
%\bibliography{myrefs}

%\end{document}
