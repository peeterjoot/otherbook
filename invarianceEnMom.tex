%
% Copyright � 2012 Peeter Joot.  All Rights Reserved.
% Licenced as described in the file LICENSE under the root directory of this GIT repository.
%

%
%
\documentclass{article}

%\input{../peeters_macros.tex}
%\input{../peeters_macros2.tex}
%\input{../peeters_macros3.tex}

\usepackage{listings}
\usepackage{txfonts} % for ointctr... (also appears to make "prettier" \int and \sum's)
% makes \grad look funny though (almost like spacegrad, but narrower)
\usepackage[bookmarks=true]{hyperref}

\usepackage{color,cite,graphicx}
   % use colour in the document, put your citations as [1-4]
   % rather than [1,2,3,4] (it looks nicer, and the extended LaTeX2e
   % graphics package.
\usepackage{latexsym,amssymb,epsf} % do not remember if these are
   % needed, but their inclusion can not do any damage

\title{Lorentz invariance of energy momentum four vector.}
\author{Peeter Joot \quad peeterjoot@pm.me }
\date{ June 21, 2009.  \(RCSfile: invarianceEnMom.tex,v \) Last \(Revision: 1.2 \) \(Date: 2009/12/03 03:24:40 \) }

\begin{document}

\maketitle{}
\tableofcontents
\section{Motivation}

A blurb on Lorentz invariance eventually removed from other notes.
Probably want to merge this with my treatment of application of the
chain rule to the wave equation as a method of
finding the Lorentz boost matrix.

\section{Prerequisite concepts.  Wave equation, and Lorentz invariance}

An unforced mechanical wave described by a function \(\psi(t,\Bx)\), propagating undamped and unforced with
velocity \(v\) is described by the familiar equation

\begin{align}\label{eqn:invarianceEnMom:waveEquationV}
\inv{v^2} \frac{\partial^2 \psi}{\partial t^2} - \spacegrad^2 \psi = 0
\end{align}

For the purposes of this discussion, a relativistic wave is described by (\(\eqnref{eqn:invarianceEnMom:waveEquationV}\)) with two
additional conditions.  The first is that the wave speed is
\(v = c\), the speed of light.  The second condition required for the label relativistic
is a restriction on the allowed coordinate transformations.  These are the linear transformations
of space time coordinates
\((t,x,y,z) \rightarrow (t', x', y', z')\) for which the wave equation retains precisely this form

\begin{align}\label{eqn:invarianceEnMom:waveEquationInv}
\inv{c^2} \frac{\partial^2 \psi}{\partial {t}^2}
- \frac{\partial^2 \psi}{\partial {x}^2}
- \frac{\partial^2 \psi}{\partial {y}^2}
- \frac{\partial^2 \psi}{\partial {z}^2}
=
\inv{c^2} \frac{\partial^2 \psi}{\partial {t'}^2}
- \frac{\partial^2 \psi}{\partial {x'}^2}
- \frac{\partial^2 \psi}{\partial {y'}^2}
- \frac{\partial^2 \psi}{\partial {z'}^2}
\end{align}

Such transformations, the Lorentz transformations,
are those that introduce no cross term such as \(\partial^2 \psi/\partial x' \partial y'\),
and do not change the wave velocity.  One can show that spatial rotations such as

\begin{align}
\begin{bmatrix}
ct' \\
x' \\
y' \\
z' \\
\end{bmatrix}
=
\begin{bmatrix}
1 & 0 & 0 & 0 \\
0 & 1 & 0 & 0 \\
0 & 0 & \cos\theta & \sin\theta \\
0 & 0 & -\sin\theta & \cos\theta \\
\end{bmatrix}
\begin{bmatrix}
ct \\
x \\
y \\
z \\
\end{bmatrix}
\end{align}

Or Lorentz boosts such as
\begin{align}
\begin{bmatrix}
ct' \\
x' \\
y' \\
z' \\
\end{bmatrix}
=
\begin{bmatrix}
\cosh\alpha & -\sinh\alpha & 0 & 0 \\
-\sinh\alpha & \cosh\alpha & 0 & 0 \\
0 & 0 & 1 & 0 \\
0 & 0 & 0 & 1 \\
\end{bmatrix}
\begin{bmatrix}
ct \\
x \\
y \\
z \\
\end{bmatrix}
\end{align}

Or any composition of such transformations meet this requirement.

In the linear transformations above the space and time coordinates were merged into a single vector representation,
the particle worldline vector, often written with shorthand such as

\begin{align}
X \equiv (ct, \Bx)
\end{align}

For such a vector, a Lorentz length can be defined

\begin{align}
X^2 \equiv c^2 t^2 - \Bx^2 \equiv c^2 t^2 - \Bx \cdot \Bx
\end{align}

Without specific discussion of the wave equation,
a more usual but equivalent definition of Lorentz transformations, are those that leave this
Lorentz length unchanged, as in

\begin{align}
c^2 t^2 - \Bx^2 = c^2 {t'}^2 - {\Bx'}^2
\end{align}

If we introduce a vector space time derivative operator

\begin{align}
\grad \equiv \left(\inv{c}\frac{\partial}{\partial t}, \spacegrad \right)
\end{align}

The Lorentz invariant length of this vector operator is in fact our wave equation operator

\begin{align}
\square \equiv \grad^2 = \inv{c^2}\frac{\partial^2}{\partial t^2} - \spacegrad \cdot \spacegrad
\end{align}

It is clear that the original requirement for wave equation invariance \eqnref{eqn:invarianceEnMom:waveEquationInv} is also contained within
this definition of Lorentz invariant length.

Unit vectors with respect to Lorentz length are necessarily Lorentz invariant.  Considering for example the time rate of change of
a particle worldline we have

\begin{align}
\left(\frac{d}{dt}(ct, \Bx) \right)^2 = c^2 - \Bv^2
\end{align}

which implies that the Lorentz length of
\begin{align}
\frac{1}{\sqrt{1 - \Bv^2/c^2}}(1, \Bv/c)
\end{align}

is just one.  For the purposes of this
Schr\"{o}dinger equation
discussion a scaling of this four vector so that it has dimensions of energy is required

\begin{align}
\frac{1}{\sqrt{1 - \Bv^2/c^2}}(m c^2, m\Bv c)
\end{align}

Algebraically, the Lorentz length of this four vector can easily be confirmed to be \((m c^2)^2\).  This quantity
we will identify as an energy-momentum four vector as follows

\begin{align}\label{eqn:invarianceEnMom:energyMomentumFourVec}
P = (E/c, \Bp)
\end{align}

With energy defined as
\begin{align}
E \equiv \frac{m c^2}{\sqrt{1 - \Bv^2/c^2}}
\end{align}

and spatial momentum defined as
\begin{align}
\Bp \equiv \frac{m \Bv}{\sqrt{1 - \Bv^2/c^2}}
\end{align}

It is beyond the scope of these notes to provide a good justification for this identification.
\footnote{This is a dodge, and having to make a statement like this shows that it is beyond the scope of the author's understanding to coherently justify this identification.  In the spirit of my engineering education I can at least work with it.}

From \eqnref{eqn:invarianceEnMom:energyMomentumFourVec} that Lorentz length of the
energy momentum four vector is

\begin{align}
P^2 = E^2/c^2 - \Bp^2 = m^2 c^2
\end{align}

\bibliographystyle{plainnat}
\bibliography{myrefs}

\end{document}
