%
% Copyright � 2012 Peeter Joot.  All Rights Reserved.
% Licenced as described in the file LICENSE under the root directory of this GIT repository.
%

%
%
%\documentclass{article}

%\input{../peeters_macros.tex}
%\input{../peeters_macros2.tex}

%\usepackage{listings}
%\usepackage{txfonts} % for ointctr... (also appears to make "prettier" \int and \sum's)
%\usepackage[bookmarks=true]{hyperref}

%\usepackage{color,cite,graphicx}
   % use colour in the document, put your citations as [1-4]
   % rather than [1,2,3,4] (it looks nicer, and the extended LaTeX2e
   % graphics package.
%\usepackage{latexsym,amssymb,epsf} % do not remember if these are
   % needed, but their inclusion can not do any damage


%\chapter{tangentspace}
\label{chap:tangentspace}
%\title{}
%\author{Peeter Joot \quad peeterjoot@protonmail.com }
\date{ March dd, 2009.  \(RCSfile: tangentspace.tex,v \) Last \(Revision: 1.9 \) \(Date: 2009/06/14 23:51:45 \) }

%\begin{document}
%\maketitle{}
%\tableofcontents
%\section{}

Your equation looks something like classical rigid body motion to me. ie:
If the position of a particle is \(x\) in a frame moving along a \(\lambda\) parametrized path \(a(\lambda)\), and that
frame is also allowed to rotate one could write something like the following for the
composite position of the particle

\begin{equation}\label{eqn:tangentspace:21}
\begin{aligned}
\overbar{x} = R(x) + a(\lambda)
\end{aligned}
\end{equation}

Or in coordinates
\begin{equation}\label{eqn:tangentspace:41}
\begin{aligned}
\overbar{x^\mu} = \Lambda^\mu_\nu x^\nu + a^\mu
\end{aligned}
\end{equation}

Derivatives with respect to the parameter (which I am thinking of as time or proper time), are then

\begin{equation}\label{eqn:tangentspace:61}
\begin{aligned}
\frac{d\overbar{x^\mu}}{d\lambda}
&=
\frac{d \Lambda^\mu_\nu}{d\lambda} x^\nu
+ \Lambda^\mu_\nu \frac{d x^\nu}{d\lambda}
+ \frac{d a^\mu}{d\lambda}
\end{aligned}
\end{equation}

As a differential displacement, one could write

\begin{equation}\label{eqn:tangentspace:81}
\begin{aligned}
d\overbar{x^\mu}
&= x^\mu(\lambda + \delta\lambda) - x^\mu(\lambda) \\
&=
d\lambda \left(
\frac{d \Lambda^\mu_\nu}{d\lambda} x^\nu
+ \Lambda^\mu_\nu \xdot^\nu
+ \adot^\mu
\right)
\end{aligned}
\end{equation}

Or
\begin{equation}\label{eqn:tangentspace:part1}
\begin{aligned}
\overbar{x}^\mu(\lambda + \delta\lambda)
&= \overbar{x}^\mu(\lambda)
+ d\lambda \frac{d \Lambda^\mu_\nu}{d\lambda} x^\nu
+ d\lambda \Lambda^\mu_\nu \xdot^\nu
+ d\lambda \adot^\mu
\end{aligned}
\end{equation}

This gives me a couple more terms than in your email
and also requires a different identification of the positions in the two frames (ie: \(\overbar{x}\), and \(x\) vectors).

In particular, I have a variation of the rotation along the worldline \(\delta \Lambda^\mu_\nu = d\lambda d\Lambda^\mu_\nu/d\lambda\) that is missing
in your equation.  Does GR give you a constraint that allows this to vanish?

Right before sending the email with this, it occurs to me that the body frame coordinates can be eliminated.  Let the inverse transformation be given by \(\Pi^\sigma_\mu\) as follows

\begin{equation}\label{eqn:tangentspace:101}
\begin{aligned}
\Pi^\sigma_\mu \Lambda^\mu_\nu = \delta^\sigma_\nu
\end{aligned}
\end{equation}

Then one can write
\begin{equation}\label{eqn:tangentspace:121}
\begin{aligned}
\Pi^\nu_\alpha(\overbar{x^\alpha} - a^\alpha)
&= \Pi^\nu_\alpha \Lambda^\alpha_\beta x^\beta  \\
&= x^\nu  \\
\end{aligned}
\end{equation}

and \eqnref{eqn:tangentspace:part1} becomes

\begin{equation}\label{eqn:tangentspace:141}
\begin{aligned}
\overbar{x}^\mu(\lambda + \delta\lambda)
&=
\overbar{x}^\mu(\lambda)
+ d\lambda \frac{d \Lambda^\mu_\nu}{d\lambda} \left( \Pi^\nu_\alpha(\overbar{x^\alpha} - a^\alpha) \right)
+ d\lambda \Lambda^\mu_\nu \xdot^\nu
+ d\lambda \adot^\mu \\
\end{aligned}
\end{equation}

\begin{equation}\label{eqn:tangentspace:partII}
\begin{aligned}
\overbar{x}^\mu(\lambda + \delta\lambda)
&=
\overbar{x}^\alpha(\lambda) \left( \delta^\mu_\alpha + d\lambda \Pi^\nu_\alpha \frac{d \Lambda^\mu_\nu}{d\lambda} \right)
- d\lambda \frac{d \Lambda^\mu_\nu}{d\lambda} a^\alpha
+ d\lambda \Lambda^\mu_\nu \xdot^\nu
+ d\lambda \adot^\mu \\
\end{aligned}
\end{equation}

Note that since \(\Pi^\sigma_\mu \Lambda^\mu_\nu = \delta^\sigma_\nu\), one has

\begin{equation}\label{eqn:tangentspace:161}
\begin{aligned}
0
&= \frac{d}{d\lambda} \left( \Pi^\sigma_\mu \Lambda^\mu_\nu \right) \\
&= \Lambda^\mu_\nu \frac{d}{d\lambda} \Pi^\sigma_\mu + \Pi^\sigma_\mu \frac{d}{d\lambda} \Lambda^\mu_\nu
\end{aligned}
\end{equation}

Believe this implies that the contracted inverse/derivative product \(\Pi^\nu_\alpha \frac{d \Lambda^\mu_\nu}{d\lambda}\) is antisymmetric.

There is still mixed coordinates in \eqnref{eqn:tangentspace:partII} (ie: both \(\overbar{x}\), and \(x\)) and I think that the final result for an incremental displacement
should likely eliminate that too (but I have not tried).

%\bibliographystyle{plainnat}
%\bibliography{myrefs}

%\end{document}
