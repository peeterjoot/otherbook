%
% Copyright � 2012 Peeter Joot.  All Rights Reserved.
% Licenced as described in the file LICENSE under the root directory of this GIT repository.
%

%
%
%\input{../peeter_prologue.tex}

\chapter{Transverse light wave propagation with z dependence}
\label{chap:maxwellHomoFirstOrder}
%\useCCL
\blogpage{http://sites.google.com/site/peeterjoot/math2009/maxwellHomoFirstOrder.pdf}
\date{Aug 9, 2009}
\revisionInfo{\(RCSfile: maxwellHomoFirstOrder.tex,v \) Last \(Revision: 1.4 \) \(Date: 2009/12/03 03:24:40 \)}

\beginArtWithToc

\section{Motivation}

Examine again the transverse and propagation direction coupling for radiation in media, without an assumption of oscillatory z dependence.  What is the coupling between the transverse and propagation direction fields with this extra degree of freedom?

SNIP.
\section{Split of the field and Maxwell equation into components}

Supposing both the transverse and propagation components of the field are all functions of all spatial dimensions

\begin{equation}\label{eqn:foo4}
\begin{aligned}
F = F(x,y,z) = \BE(x,y,z) + I\BB(x,y,z)/\sqrt{\mu\epsilon}
\end{aligned}
\end{equation}

even without something that is amenable to simple separation by variables into \(x,y\) and \(z\) parts, we can still geometrically decompose the field into transverse plane and propagation components.  That is

\begin{equation}\label{eqn:foo5}
\begin{aligned}
F_z(x,y,z) &\equiv \inv{2}(F + \zcap F \zcap) \\
F_t(x,y,z) &\equiv \inv{2}(F - \zcap F \zcap)
\end{aligned}
\end{equation}

Noting that \(\zcap^2 = 1\) it is readily verified that \(F_t \zcap = - \zcap F_t\), and \(F_z \zcap = \zcap F_z\).  These anticommutation and commutation properties also hold for the transverse and propagation components of the gradient as well.

Forming the sum \(\spacegrad F \zcap + \zcap \spacegrad F = \lambda F \zcap + \zcap \lambda F\) we have for the LHS

\begin{equation}\label{eqn:maxwellHomoFirstOrder:28}
\begin{aligned}
\spacegrad F \zcap + \zcap \spacegrad F
&= (\spacegrad_t + \spacegrad_z) (F_t + F_z) \zcap + \zcap (\spacegrad_t + \spacegrad_z) (F_t + F_z)  \\
&= (\spacegrad_t + \spacegrad_z) (F_t + F_z) \zcap + (-\spacegrad_t + \spacegrad_z) (-F_t + F_z) \zcap  \\
&= 2 \left( \spacegrad_t F_t + \spacegrad_z F_z \right) \zcap
\end{aligned}
\end{equation}

For our choice to use the transverse plane bivector as the imaginary \(i = \xcap\ycap\) we have \(\zcap i = i \zcap\), and thus \(\zcap \lambda = \lambda \zcap\).  Therefore for the RHS we have

\begin{equation}\label{eqn:maxwellHomoFirstOrder:48}
\begin{aligned}
\lambda F \zcap + \zcap \lambda F
&=
\lambda (F_t + F_z) \zcap + \zcap \lambda (F_t + F_z) \\
&=
\lambda (F_t + F_z) \zcap + \lambda (-F_t + F_z) \zcap \\
\end{aligned}
\end{equation}

Maxwell's equation in terms of transverse and propagation components is thus

\begin{equation}\label{eqn:foo6}
\begin{aligned}
\spacegrad_t F_t &= (\lambda -\spacegrad_z) F_z
\end{aligned}
\end{equation}

A left multiplication by \(-\zcap\) gives us

\begin{equation}\label{eqn:foo7}
\begin{aligned}
\left(\partial_z - I\sqrt{\mu\epsilon}\frac{\omega}{c} \right) F_z &= \zcap \spacegrad_t F_t
\end{aligned}
\end{equation}

We want to solve for \(F_z\), and have reduced the problem to that of a first order non-homogeneous linear differential equation of the form

\begin{equation}\label{eqn:foo8}
\begin{aligned}
u' - I \alpha u = g
\end{aligned}
\end{equation}

Some additional care solving this is required since our functions are multivector valued, so we cannot necessarily assume commutative behavior.

\EndArticle
