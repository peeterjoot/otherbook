%
% Copyright � 2012 Peeter Joot.  All Rights Reserved.
% Licenced as described in the file LICENSE under the root directory of this GIT repository.
%

%
%
%\documentclass{article}

%\input{../peeters_macros.tex}
%\input{../peeters_macros2.tex}

%\usepackage[bookmarks=true]{hyperref}

%\usepackage{color,cite,graphicx}
%   % use colour in the document, put your citations as [1-4]
%   % rather than [1,2,3,4] (it looks nicer, and the extended LaTeX2e
%   % graphics package.
%\usepackage{latexsym,amssymb,epsf} % do not remember if these are
%   % needed, but their inclusion can not do any damage


%\chapter{Chapter I-III comments on Alan's book}
\label{chap:reviewAlanGabook}
%\author{Peeter Joot \quad peeterjoot@protonmail.com}
\date{ Mmm dd, 2009.  \(RCSfile: reviewAlanGabook.tex,v \) Last \(Revision: 1.8 \) \(Date: 2009/06/14 23:51:45 \) }

%\begin{document}

%\maketitle{}
%\tableofcontents

\section{Chapter I}

\subsection{page 10. vectors}

More accurate or at least more general would be the use of
momentum instead of velocity in the 6n vector, but I am
guessing this simplification has been made on purpose.


For \(n \approx 10^{23}\) you have not specified the size of the box.

\section{Chapter 3}

\subsection{page 35. matrices, properties and non-properties}

If I was a new learner of the subject I would have been uncomfortable with
the omission
of the associativity proof where you say "The proofs use uninteresting juggling of the indices".  While I agree that including this proof inline would detract
from the flow of things, I had be personally inclined to make this be
something more.  Perhaps a problem where you show how to do half of it:

\begin{equation}\label{eqn:reviewAlanGabook:20}
\begin{aligned}
(AB) C
&=
\begin{bmatrix}
a_{ij}
\end{bmatrix}
\begin{bmatrix}
b_{ij}
\end{bmatrix} C \\
&=
\begin{bmatrix}
\sum_k a_{ik} b_{kj}
\end{bmatrix} C \\
\end{aligned}
\end{equation}

Let \(\sum_k a_{ik} b_{kj} = d_{ij}\)

\begin{equation}\label{eqn:reviewAlanGabook:40}
\begin{aligned}
\implies
(AB) C
&=
\begin{bmatrix}
d_{ij}
\end{bmatrix}
\begin{bmatrix}
c_{ij}
\end{bmatrix} \\
&=
\begin{bmatrix}
\sum_k d_{ik} c_{kj}
\end{bmatrix} \\
&=
\begin{bmatrix}
\sum_{m,k} a_{im} b_{mk} c_{kj}
\end{bmatrix} \\
\end{aligned}
\end{equation}

Then make the expansion of the other grouping as the remainder of the exercise.

\subsection{Page 37.  ex 3.8, 3.9}

Inline matrices missing brackets.

\subsection{Page 38.  theorem 3.7}

Looks funny without qualification since one immediately thinks of left and
right inverses.

\begin{equation}\label{eqn:reviewAlanGabook:60}
\begin{aligned}
A =
\begin{bmatrix}
1 & 0 & 0 \\
0 & 0 & 1 \\
\end{bmatrix}
\end{aligned}
\end{equation}

\begin{equation}\label{eqn:reviewAlanGabook:80}
\begin{aligned}
B =
\begin{bmatrix}
1 & 0 \\
0 & 0 \\
0 & 1 \\
\end{bmatrix}
\end{aligned}
\end{equation}

\begin{equation}\label{eqn:reviewAlanGabook:100}
\begin{aligned}
A B = I_2
\end{aligned}
\end{equation}

ie: B is right inverse.

\subsection{page 39}

I think your notation for transpose will confuse people when they move on to other studies, since asterisk is much more normally used for Hermitian transpose,
while a text T is used for real numbers.  I had restrict this discussion to real
numbers for clarity and not use a symbol that implies more.

\subsection{page 41. change of basis problem}

If this were actually a readers first introduction to matrix math, then
if you punt change of basis to a problem like this, then without elaboration
I do not think there is much hope that the reader will be able to get much out
of this.  Either omit it, or elaborate (as a non-problem) IMO.

\subsection{page 42. systems of equations}

You mention that due to potential scope of the problem, systematic and automatic computed methods are required, but do not mention the other important aspect of this.  The traditional row reduction methods taught in a first year course have massive numerical stability issues since they omit pivots and do not touch on how to deal with near zero values.  Since you are systematically omitting the row reduction coverage in this text, I think it would be good to give a good example of how a naive solution would produce whacked answers.

In an exercise give a super quickly description of the row reduction method works and have the student apply it to something ill formed like

\begin{equation}\label{eqn:reviewAlanGabook:120}
\begin{aligned}
\begin{bmatrix}
1 & 1 \\
10^{-23} & 0 \\
\end{bmatrix}
\begin{bmatrix}
x \\
y \\
\end{bmatrix}
=
\begin{bmatrix}
1 \\
2 \\
\end{bmatrix}
\end{aligned}
\end{equation}

(or perhaps some other better example(s) from a numerical analysis text).

%\bibliographystyle{plainnat}
%\bibliography{myrefs}

%\end{document}
