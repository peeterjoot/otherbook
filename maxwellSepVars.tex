%
% Copyright � 2012 Peeter Joot.  All Rights Reserved.
% Licenced as described in the file LICENSE under the root directory of this GIT repository.
%

%
%
%\input{../peeter_prologue.tex}

\chapter{Separation of variables applied to homogeneous Maxwell equation}
\label{chap:maxwellSepVars}
%\useCCL
\blogpage{http://sites.google.com/site/peeterjoot/math2009/maxwellSepVars.pdf}
\date{Aug 9, 2009}
\revisionInfo{\(RCSfile: maxwellSepVars.tex,v \) Last \(Revision: 1.5 \) \(Date: 2009/12/03 03:24:40 \)}

\beginArtWithToc

\section{Motivation}

Apply separation of variables to the first order and second order variations of Maxwell's equation.  Consider reflection and penetration.

\section{Setup}

Again following Jackson \citep{jackson1975cew}, we use CGS units.  The first and second order Maxwell's equation in these units, with \(F = \BE + I\BB/\sqrt{\mu\epsilon}\) are, respectively

\begin{equation}\label{eqn:foo1}
\begin{aligned}
0 &= (\spacegrad + \sqrt{\mu\epsilon} \partial_0) F \\
0 &= (\spacegrad^2 - {\mu\epsilon} \partial_{00}) F
\end{aligned}
\end{equation}

We can apply separation of variables to either.  Let us write \(F = \calF T\).  For the first order equation we have

\begin{equation}\label{eqn:maxwellSepVars:25}
\begin{aligned}
(\spacegrad \calF) T + \frac{\sqrt{\mu\epsilon}}{c} \calF T' = 0
\end{aligned}
\end{equation}

Separating, with left and right multiplication by \(\calF\) and \(T\) respectively, we have

\begin{equation}\label{eqn:foo2}
\begin{aligned}
\inv{\calF} (\spacegrad \calF) = -\frac{\sqrt{\mu\epsilon}}{c} T' \inv{T} = \text{constant}
\end{aligned}
\end{equation}

For the constant let us write \(A \sqrt{\mu\epsilon} \omega/c\), we then have

\begin{equation}\label{eqn:foo5}
\begin{aligned}
T' &= -\omega A T \\
\spacegrad F &= F A \sqrt{\mu\epsilon} \frac{\omega}{c}
\end{aligned}
\end{equation}

The factor \(A\) probably needs to be multivector valued.  Let us defer thinking about this till after examination of the second order equation.  There we have

\begin{equation}\label{eqn:maxwellSepVars:45}
\begin{aligned}
(\spacegrad^2 \calF) T - \frac{\mu\epsilon}{c^2} \calF T'' = 0
\end{aligned}
\end{equation}

So can separate as

\begin{equation}\label{eqn:foo3}
\begin{aligned}
\inv{\calF} (\spacegrad^2 \calF) = \frac{\mu\epsilon}{c^2} T'' \inv{T} = \text{constant}
\end{aligned}
\end{equation}

A choice to make the constant negative in \eqnref{eqn:foo3} results in sinusoidal time dependence, and in fact produces the Helmholtz equation.  That is

\begin{equation}\label{eqn:foo4}
\begin{aligned}
T'' &= - \omega^2 T \\
  0 &= \left(\spacegrad^2 + \mu\epsilon \frac{\omega^2}{c^2} \right) \calF
\end{aligned}
\end{equation}

The neat thing about this equation is that this has exactly the same form as the time independent Schrodinger equation, so the techniques used for calculating barrier penetration and reflection (for example \citep{bohm1989qt} can be borrowed.  In the quantum equation calculation of current density when the potential changed was of interest.  It is probably energy or momentum density that is of interest for this optics case.

Observe that we have a lot more flexibility in picking the imaginary than in scalar calculus.  Possible solutions for the time dependence are

\begin{equation}\label{eqn:maxwellSepVars:65}
\begin{aligned}
T &\propto e^{\pm I\omega t} \\
T &\propto e^{\pm i\omega t} \\
T &\propto e^{\pm I \kcap \omega t}
\end{aligned}
\end{equation}

The first uses the spatial pseudoscalar \(I = \sigma_1 \sigma_2 \sigma_3\).  The second uses a regular non-geometric imaginary, and if we make that choice we will also later take real parts of the eventual solution \(F = \calF T\) to get a physical solution.  Finally, we have the choice to use any unit bivector, here parametrized by duality using the perpendicular to the plane direction \(\kcap\).

\EndArticle
