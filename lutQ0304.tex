%
% Copyright � 2012 Peeter Joot.  All Rights Reserved.
% Licenced as described in the file LICENSE under the root directory of this GIT repository.
%

%
%
%\documentclass{article}

%\input{../peeters_macros.tex}
%\input{../peeters_macros2.tex}

%\usepackage[bookmarks=true]{hyperref}

%\usepackage{color,cite,graphicx}
   % use colour in the document, put your citations as [1-4]
   % rather than [1,2,3,4] (it looks nicer, and the extended LaTeX2e
   % graphics package.
%\usepackage{latexsym,amssymb,epsf} % do not remember if these are
   % needed, but their inclusion can not do any damage


%\chapter{lut q0304}
\label{chap:lutQ0304}
%\author{Peeter Joot \quad peeterjoot@protonmail.com }
\date{ March dd, 2009.  \(RCSfile: lutQ0304.tex,v \) Last \(Revision: 1.5 \) \(Date: 2009/06/14 23:51:45 \) }

%\begin{document}

Attempt to decode a fragment of
\href{http://en.wikipedia.org/wiki/Frame_fields_in_general_relativity}{Static observers in Schwartzchild vacuum, fro the Frame field article.}

%==Example: Static observers in Schwarzschild vacuum==
%
%It will be instructive to consider in some detail a few simple examples. Consider the famous [[Schwarzschild metric|Schwarzschild vacuum]] that models spacetime outside an isolated nonspinning spherically symmetric massive object, such as a star. In most textbooks one finds the metric tensor written in terms of a static polar spherical chart, as follows:
%:<math>ds^2 = -(1-2m/r) \, dt^2 + \frac{dr^2}{1-2m/r} + r^2 \, \left( d\theta^2 + \sin(\theta)^2 \, d\phi^2 \right)</math>
%:<math> -\infty < t < \infty, \; 2 m < r < \infty, \; 0 < \theta < \pi, \; -\pi < \phi < \pi</math>
%More formally, the metric tensor can be expanded with respect to the coordinate cobasis as
%:<math>g = -(1-2m/r) \, dt \otimes dt + \frac{1}{1-2m/r} \, dr \otimes dr + r^2 \, d\theta \otimes d\theta + r^2 \sin(\theta)^2 \, d\phi \otimes d\phi</math>
A coframe can be read off from this expression:
%:<math>
\begin{equation}\label{eqn:lutQ0304:20}
\begin{aligned}
\sigma^0 = -\sqrt{1-2m/r} \, dt, \;
\sigma^1 = \frac{dr}{\sqrt{1-2m/r}}, \;
\sigma^2 = r d\theta, \;
\sigma^3 = r \sin(\theta) d\phi
\end{aligned}
\end{equation}
%</math>
%To see that this coframe really does correspond to the Schwarzschild metric tensor, just plug this coframe into
%:<math>g = -\sigma^0 \otimes \sigma^0 + \sigma^1 \otimes \sigma^1 + \sigma^2 \otimes \sigma^2 + \sigma^3 \otimes \sigma^3</math>
%
The frame dual to the coframe is
%:<math>
\begin{equation}\label{eqn:lutQ0304:40}
\begin{aligned}
\vec{e}_0 = \frac{1}{\sqrt{1-2m/r}} \partial_t, \;
\vec{e}_1 = \sqrt{1-2m/r} \partial_r, \;
\vec{e}_2 = \frac{1}{r} \partial_\theta, \;
\vec{e}_3 = \frac{1}{r \sin(\theta)} \partial_\phi
\end{aligned}
\end{equation}
%</math>
%(The minus sign on <math>\sigma^0</math> ensures that <math>\vec{e}_0</math> is ''future pointing''.) This is the frame that models the experience of '''static observers''' who use rocket engines to ''"hover" over the massive object''.
The thrust they require to maintain their position is given by the magnitude of the acceleration vector
%:<math>
\begin{equation}\label{eqn:lutQ0304:60}
\begin{aligned}
\nabla_{\vec{e}_0} \vec{e}_0
&= \frac{m/r^2}{\sqrt{1-2m/r}} \, \vec{e}_1  \\
&= {m/r^2} \partial_r
\end{aligned}
\end{equation}
%</math>

What is the gradient notation mean?  Is it the gradient from Doran/Lasenby?  There for a vector in a basis \(\vec{e}_\mu\), or the dual basis \(\sigma^\mu\) is

\begin{equation}\label{eqn:lutQ0304:80}
\begin{aligned}
x
&= x_\mu \sigma^\mu \\
&= x^\mu \vec{e}_\mu \\
\end{aligned}
\end{equation}

The gradient can be expressed in either the basis or its dual
\begin{equation}\label{eqn:lutQ0304:100}
\begin{aligned}
\grad
&= \sum_\mu \sigma^\mu \frac{\partial}{\partial x^\mu} \equiv \sigma^\mu \partial_\mu \\
&= \sum_\mu \vec{e}_\mu \frac{\partial}{\partial x_\mu} \equiv \vec{e}_\mu \partial_\mu
\end{aligned}
\end{equation}

Perhaps such a gradient is related to this GR frame gradient?  Let us try applying that to \(\vec{e}_0\)
\begin{equation}\label{eqn:lutQ0304:120}
\begin{aligned}
\grad \vec{e}_0
&= \sigma^\mu \partial_\mu \vec{e}_0 \\
&=
\left( -\sqrt{1-2m/r} \partial_t
+\frac{dr}{\sqrt{1-2m/r}} \partial_r
+r d\theta \partial_\theta
+r \sin(\theta) d\phi \partial_\phi \right) \frac{1}{\sqrt{1-2m/r}} \partial_t \\
&=
\frac{dr}{\sqrt{1-2m/r}} \partial_r \frac{1}{\sqrt{1-2m/r}} \partial_t \\
&=
\frac{dr}{\sqrt{1-2m/r}} (-1/2) \frac{1}{(\sqrt{1-2m/r})^3} (-2m)(-1/r^2) \partial_t \\
&=
\frac{dr}{\sqrt{1-2m/r}} (-1) \frac{1}{(\sqrt{1-2m/r})^3} (m)(1/r^2) \partial_t \\
&=
-\frac{(m/r^2)dr}{(1-2m/r)^2} \partial_t \\
&=
-\sigma^1 \frac{(m/r^2)}{(1-2m/r)^{3/2}} \partial_t \\
&=
-\sigma^1 \frac{(m/r^2)}{(1-2m/r)} \vec{e}_0 \\
\end{aligned}
\end{equation}

This has got the \(m/r^2\), but also an extra term in the denominator

%(which is justifiable in a GA context by vectorizing the Euler-Lagrange equations)

%This is radially outward pointing, since the observers need to accelerate ''away'' from the object to avoid falling toward it. On the other hand, the spatially projected Fermi derivatives of the spatial basis vectors (with respect to <math>\vec{e}_0</math>) vanish, so this is a nonspinning frame.
%
%The components of various tensorial quantities with respect to our frame and its dual coframe can now be computed.
%
%For example, the [[electrogravitic tensor|tidal tensor]] for our static observers is defined using tensor notation (for a coordinate basis) as
%:<math> E[X]_{ab} = R_{ambn} \, X^m \, X^n</math>
%where we write <math>\vec{X} = \vec{e}_0 </math> to avoid cluttering the notation. Its only non-zero components with respect to our coframe turn out to be
%:<math> E[X]_{11} = -2m/r^3, \; E[X]_{22} = E[X]_{33} = m/r^3</math>
%The corresponding coordinate basis components are
%:<math> E[X]_{rr} = -2m/r^2/(1-2m/r), \; E[X]_{\theta \theta} = m/r, \; E[X]_{\phi \phi} = m \sin(\theta)^2/r</math>
%
%(A quick note concerning notation: many authors put [[caret]]s over ''abstract'' indices referring to a frame. When writing down ''specific components'', it is convenient to denote frame components by 0,1,2,3 and coordinate components by <math>t,r,\theta,\phi</math>. Since an expression like <math>S_{ab} = 36 m/r</math> does not make sense as a [[tensor equation]], there should be no possibility of confusion.)
%
%Compare the [[tidal tensor]] <math>\Phi</math> of Newtonian gravity, which is the '''[[traceless]] part''' of the [[Hessian matrix|Hessian]] of the gravitational potential <math>U</math>. Using tensor notation for a tensor field defined on three-dimensional euclidean space, this can be written
%:<math>\Phi_{ij} = U_{,i j} - \frac{1}{3} {U^{,k}}_{,k} \, \eta_{ij} </math>
%The reader may wish to crank this through (notice that the trace term actually vanishes identically when U is harmonic) and compare results with the following elementary approach:
%we can compare the gravitational forces on two nearby observers lying on the same radial line:
%:<math> m/(r+h)^2 - m/r^2 = -2m/r^3 \, h + 3m/r^4 \, h^2 + O(h^3) </math>
%Because in discussing tensors we are dealing with [[multilinear algebra]], we retain only first order terms, so <math>\Phi_{11} = -2m/r^3</math>. Similarly, we can compare the gravitational force on two nearby observers lying on the same sphere <math>r = r_0</math>. Using some elementary trigonometry and the small angle approximation, we find that the force vectors differ by a vector tangent to the sphere which has magnitude
%:<math> \frac{m}{r_0^2} \, \sin(\theta) \approx \frac{m}{r_0^2} \, \frac{h}{r_0} = \frac{m}{r_0^3} \, h</math>
%By using the small angle approximation, we have ignored all terms of order <math>O(h^2)</math>, so the tangential components are <math>\Phi_{22} = \Phi_{33} = m/r^3</math>. Here, we are referring to the obvious frame obtained from the polar spherical chart for our three-dimensional euclidean space:
%:<math> \vec{\epsilon}_1 = \partial_r, \; \vec{\epsilon}_2 = \frac{1}{r} \, \partial_\theta, \; \vec{\epsilon}_3 = \frac{1}{r \sin \theta} \, \partial_\phi</math>
%
%Plainly, the coordinate components <math>E[X]_{\theta \theta}, \, E[X]_{\phi \phi}</math> computed above do not even scale the right way, so they clearly cannot correspond to what an observer will measure even approximately. (By coincidence, the Newtonian tidal tensor components agree exactly with the relativistic tidal tensor components we wrote out above.)
%

%\maketitle{}
%\tableofcontents
%\section{}

%\bibliographystyle{plainnat}
%\bibliography{myrefs}

%\end{document}
