%
% Copyright � 2012 Peeter Joot.  All Rights Reserved.
% Licenced as described in the file LICENSE under the root directory of this GIT repository.
%

%
%
%\input{../peeter_prologue.tex}

%\chapter{Epsilon Mixed Tensor}
\label{chap:epsilonMixed}
\date{July 14, 2009}
\revisionInfo{\(RCSfile: epsilonMixed.tex,v \) Last \(Revision: 1.2 \) \(Date: 2009/12/03 03:24:40 \)}

\beginArtWithToc

\section{Motivation}

Continuing with the attempt to understand how infinitesimal Poincare
symmetries generate the electrodynamic (symmetric) stress energy
tensor, examine symmetries associated with infinitesimal rotations
or boosts (ignoring the translational aspect of the Poincare transformation).

\section{Epsilon Mixed Tensor from incremental rotation or boost}

Write the Lorentz transform of a vector through a small angle or rapidity \(d\theta\)

\begin{equation}\label{eqn:epsilonMixed:20}
\begin{aligned}
x &\rightarrow x' = R x \tilde{R} \\
R &= e^{I d\theta/2}
\end{aligned}
\end{equation}

Neglecting all but first order terms we have

\begin{equation}\label{eqn:epsilonMixed:40}
\begin{aligned}
x'
&\approx (1 + Id\theta/2)x(1 - Id\theta/2) \\
&\approx x + \frac{d\theta}{2}(I x + x I) \\
\end{aligned}
\end{equation}

\begin{equation}\label{eqn:incremental}
\begin{aligned}
x' &= x + {d\theta} I \cdot x
\end{aligned}
\end{equation}

Here \(I\) has been assumed to be a bivector for the space in question.  In Minkowski space for a boost this could be for example the spacetime plane \(\gamma_1 \gamma_0\), or a spatial plane for a rotation \(I = \gamma_1 \gamma_2\).  Ignoring the details of the vector space for now, just assume a pair of reciprocal frames \(\{e^k\}\), \(\{e_k\}\), both spanning the space and mutually pairwise
normal

\begin{equation}\label{eqn:epsilonMixed:60}
\begin{aligned}
e^k \cdot e_j &= {\delta^k}_j \\
x &= x^k e_k = x_k e^k
\end{aligned}
\end{equation}

Without any loss of generality select two vectors \(e_u\) and \(e_v\), within the rotational plane

\begin{equation}\label{eqn:epsilonMixed:80}
\begin{aligned}
I &= e_u \wedge e_v
\end{aligned}
\end{equation}

Expansion of \eqnref{eqn:incremental} by dotting with \(e^c\) now yields the
antisymmetric tensor equivalent, \(\epsilon^{a c} = -\epsilon^{c a}\)

\begin{equation}\label{eqn:epsilonMixed:100}
\begin{aligned}
{x^c}'
&= x^c + {d\theta} (e_u \wedge e_v) \cdot (x_a e^a) \cdot e^c \\
&= x^c + {d\theta} x_a ( e_u {\delta_v}^a -e_v {\delta_u}^a ) \cdot e^c \\
&= x^c + {d\theta} x_a ( {\delta_u}^c {\delta_v}^a - {\delta_v}^c {\delta_u}^a )
\end{aligned}
\end{equation}

i.e.

\begin{equation}\label{eqn:epsilonMixed:120}
\begin{aligned}
\epsilon^{a c} &\equiv d\theta ({\delta_u}^c {\delta_v}^a - {\delta_v}^c {\delta_u}^a )  \\
{x^c}' &= x^c + x_a \epsilon^{a c}
\end{aligned}
\end{equation}

To translate to both coordinate indices up in this (not necessarily orthonormal) frame take
dot products

\begin{equation}\label{eqn:epsilonMixed:140}
\begin{aligned}
x_a = x \cdot e_a = (x_b e^b) \cdot e_a = (x^d e_d) \cdot e_a
\end{aligned}
\end{equation}

The dot product is the (symmetric) metric tensor \(e_a \cdot e_d = \eta_{a d}\), so we can
write for \eqnref{eqn:incremental}

\begin{equation}\label{eqn:epsilonMixed:160}
\begin{aligned}
{x^c}' &= x^c + x^d {\epsilon_{d}}^c \\
{\epsilon_d}^c
&= \eta_{a d} \epsilon^{a c} \\
&= d\theta \eta_{a d} ({\delta_u}^c {\delta_v}^a - {\delta_v}^c {\delta_u}^a )  \\
\end{aligned}
\end{equation}

Need inverse relation.

\EndArticle
