%
% Copyright � 2012 Peeter Joot.  All Rights Reserved.
% Licenced as described in the file LICENSE under the root directory of this GIT repository.
%

%
%
%\documentclass{article}      % Specifies the document class

%\input{../peeters_macros.tex}
%
% The real thing:
%

                             % The preamble begins here.
%\chapter{Questioning equation 8.8 from GASC} % Declares the document's title.
\label{chap:gacsQ88}
%\author{Peeter Joot \quad peeterjoot@protonmail.com}         % Declares the author's name.
\date{ May 9, 2008.  \(RCSfile: gacsQ88.tex,v \) Last \(Revision: 1.9 \) \(Date: 2009/06/14 23:51:45 \) }

%\begin{document}             % End of preamble and beginning of text.

%\maketitle{}

\section{email 1}

In formula 8.8 you take the scalar derivative of a rotor applied to a blade, where ``\(I \phi\) is a function of \(\tau\) so that both rotation plane and rotation angle may vary''.  I am not sure about the end result, but one of the intermediate steps looks wrong to me.  It appears that you implicitly use the following to arrive at your second line:

\begin{equation}\label{eqn:gacsQ88:20}
\Exp{ I \phi }' = ( I \phi )' \Exp{ I \phi }
\end{equation}

(writing primes for your \(\partial_\tau\)).

If I calculate a bivector exponential I get:

\begin{equation}\label{eqn:gacsQ88:180}
\begin{aligned}
\Exp{ I \phi }'
&= (\Cos{\phi} + I \Sin{\phi})' \\
&= (-\Sin{\phi} + I \Cos{\phi}) \phi' + I' \Sin{\phi} \\
&= I \phi' \Exp{ I \phi } + I' \Sin{\phi} \\
\end{aligned}
\end{equation}

I believe this is generally only \((I \phi)' \Exp{ I \phi }\) when \(I' = 0\), which contradicts the description where you allow for variation of the rotation plane?
%  If I misunderstand I had appreciate if you could point out how (and consider this point for clarification in a future edition).  Thanks!

\section{email 2}

I have answered my own question about what equation 8.8 should look like if one allows for a general rotation (ie: allowing for wobble), as implied by the text.

Applying logic from the rigid body treatment in a book like ``GA for Physicists'' then you can arrive at a commutator result for blades similar to your equation 8.8:

\begin{equation}\label{eqn:gacsQ88:40}
X = R X_0 R^\dagger
\end{equation}
\begin{equation}\label{eqn:gacsQ88:200}
\begin{aligned}
\implies
X' &= R' X_0 R^\dagger + R X_0 {R'}^\dagger \\
   &= R' (R^\dagger R) X_0 R^\dagger + R X_0 (R^\dagger R) {R'}^\dagger \\
   &= R' R^\dagger X + X R {R'}^\dagger \\
\end{aligned}
\end{equation}

Until this point the procedure is not that much different than your 8.8 treatment, but to get a commutator result, you need the rigid body trick:

\begin{equation}\label{eqn:gacsQ88:60}
(R R^\dagger)' = 1' = 0 = R {R'}^\dagger + R' R^\dagger
\end{equation}

That trick you can also find in non-GA rigid body texts with matrices \(R R^T = I\).  The rotor derivative will also only have 0,2 grades, so the product potentially has 0,2,4 grades.  Since it value is the negation of its reverse, it must be strictly grade 2.  Introducing \(R' R^\dagger = \inv{2} \Omega\), and using the result above, this gives:

\begin{equation}\label{eqn:gacsQ88:80}
X' = \inv{2}(\Omega X - X \Omega) = \Omega \times X
\end{equation}

%Hindsight shows that incorporating the factor of two in \(\Omega\) would be cleaner, but that is sneaky:)

I am pretty sure that this is how 8.8 has to look if you allow for wobbly rotation (\(\frac{dI}{d\tau} \ne 0\)), so the question part of my original note is resolved to my satisfaction.  Writing this up, it occurs to me that, that your equation 8.8, while exact for a fixed orientation rotation, is probably accurate up to a first order differential approximation of a wobbly rotation.

\section{Post emails}

Let us try this.  With the derivative calculated above we have for \(R'R^\dagger\), where \(R = \Exp{-I \halfPhi}\) :

\begin{equation}\label{eqn:gacsQ88:220}
\begin{aligned}
R'R^\dagger
&= \left(-I \phi'/2 \Exp{ -I \halfPhi } - I' \Sin{\halfPhi} \right)\Exp{ I \halfPhi } \\
&= -I \phi'/2 - I' \Sin{\halfPhi} \Exp{ I \halfPhi } \\
&= -I \phi'/2 - I' \Sin{\halfPhi}\Cos{\halfPhi} + I'I \Sin{\halfPhi} \\
&= -I \phi'/2 - I'/2 \Sin{\phi} + I'I \Sin{\halfPhi} \\
\end{aligned}
\end{equation}

\begin{equation}\label{eqn:gacsQ88:100}
\implies
\Omega = -I \phi' - I' \Sin{\phi} + 2 I'I \Sin{\halfPhi}
\end{equation}
\begin{equation}\label{eqn:gacsQ88:240}
\begin{aligned}
X'
&= \left( -I \phi' - I' \Sin{\phi} + 2 I'I \Sin{\halfPhi} \right) \times X \\
&= \inv{2} \left( \left( -I \phi' - I' \Sin{\phi} + 2 I'I \Sin{\halfPhi} \right) X - X \left( -I \phi' - I' \Sin{\phi} + 2 I'I \Sin{\halfPhi} \right) \right) \\
&= -I \phi' \times X  - I'\Sin{\phi} \times X + \Sin{\halfPhi} \left( I'I X - X I' I \right)  \\
&= X \times \left(I \phi' + I'\Sin{\phi} \right ) + 2 \Sin{\halfPhi} \left( I'I \right) \times X \\
&= X \times \left(I \phi' + I'\Sin{\phi} - 2 \Sin{\halfPhi} I'I \right) \\
\end{aligned}
\end{equation}

Now for very small \(\phi\), \(\Sin{\phi} = \phi\), so we have:

\begin{equation}\label{eqn:gacsQ88:120}
X' \approx X \times (I \phi)' + \phi ( I'I ) \times X
\end{equation}
or:
\begin{equation}\label{eqn:gacsQ88:140}
X' \approx X \times \left( (I \phi)' - \phi ( I'I ) \right)
\end{equation}

In the GASC notation this is
\begin{equation}\label{eqn:gacsQ88:160}
\partial_\tau X \approx X \times \partial_\tau (I \phi) - X \times (\partial_\tau(I) I \phi)
\end{equation}

The first part of this is equation 8.8, but only holds when the angle \(\phi\) is small.  We also have a corrective term that comes only from variation in the bivector orientation, and is also an approximation that holds only when the angle \(\phi\) is small.

%\end{document}               % End of document.
