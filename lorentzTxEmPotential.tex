%
% Copyright � 2012 Peeter Joot.  All Rights Reserved.
% Licenced as described in the file LICENSE under the root directory of this GIT repository.
%

%
%
%\documentclass{article}

%\input{../peeters_macros.tex}
%\input{../peeters_macros2.tex}

%\usepackage{listings}
%\usepackage{txfonts} % for ointctr... (also appears to make "prettier" \int and \sum's)

%\usepackage[bookmarks=true]{hyperref}

%\chapter{Lorentz force interaction} % Declares the document's title.
\label{chap:lorentzTxEmPotential}
%\author{Peeter Joot \quad peeterjoot@protonmail.com}
\date{ October 17, 2008.  \(RCSfile: lorentzTxEmPotential.tex,v \) Last \(Revision: 1.12 \) \(Date: 2009/06/14 23:51:45 \) }

%\begin{document}

%\maketitle{}
%\tableofcontents

\section{Motivation}

Explore the relationships between proper velocity and current density.

\citep{goldstein1951cm} has a combined Lagrangian for the field and Lorentz force:

\begin{equation}\label{eqn:lorentzTxEmPotential:20}
\begin{aligned}
\LL = \int \left\{ \frac{E^2 - B^2}{8 \pi} - \sum_i q_i \delta(\Br-\Br_i)\left(\phi - \Bv_i/c \cdot \BA\right)\right\} dV + \inv{2} m_i \Bv_i^2
\end{aligned}
\end{equation}

Where alternate variation of the field, or coordinates produces the field equations or Lorentz equation respectively.

Compare to the covariant Lagrangian for the field and interaction (for metric \(+---\))

\begin{equation}\label{eqn:lorentz_tx_em_potential:maxlag}
\begin{aligned}
\LL_{\text{field}} &= -\frac{\epsilon_0}{2} (\grad \wedge A)^2 + A \cdot J/c \\
\LL_{\text{interaction}} &= \inv{2} m v^2 + q A \cdot (v/c)
\end{aligned}
\end{equation}

(SI units this time).

These can be shown to produce Maxwell's equation and the Lorentz force equation respectively
\begin{equation}\label{eqn:lorentzTxEmPotential:40}
\begin{aligned}
\grad F &= J/c\epsilon_0 \\
\pdot &= q F \cdot (v/c)
\end{aligned}
\end{equation}

Derivations of these can be found in \citep{classicalmechanics:PJFieldLagrangian}, and \citep{classicalmechanics:PJSrLorentzForce} respectively.

The final aim is
find how to relate the \(A\cdot J\), and \(q A \cdot v\) interaction quantities in covariant field and
Lorentz Lagrangians.  Intuition tells me that considering the Lorentz transformation of \(J\) and \(A\) may help
shed some light on this.

Some four potentials discussion perhaps useful as background can be found in \citep{gabookII:PJFourPotential}.

%\bibliographystyle{plainnat}
%\bibliography{myrefs}

%\end{document}
