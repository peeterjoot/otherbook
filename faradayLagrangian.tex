%
% Copyright � 2012 Peeter Joot.  All Rights Reserved.
% Licenced as described in the file LICENSE under the root directory of this GIT repository.
%

%
%
%\documentclass{article}

%\input{../peeters_macros.tex}
%\input{../peeters_macros2.tex}

%\usepackage[bookmarks=true]{hyperref}

%\usepackage{color,cite,graphicx}
   % use colour in the document, put your citations as [1-4]
   % rather than [1,2,3,4] (it looks nicer, and the extended LaTeX2e
   % graphics package.
%\usepackage{latexsym,amssymb,epsf} % do not remember if these are
   % needed, but their inclusion can not do any damage


\chapter{REMOVED FROM electric field energy}
\label{chap:faradayLagrangian}
%\author{Peeter Joot \quad peeterjoot@protonmail.com}
\date{ Feb dd, 2009.  \(RCSfile: faradayLagrangian.tex,v \) Last \(Revision: 1.8 \) \(Date: 2009/06/14 23:51:45 \) }

%\begin{document}

%\maketitle{}

%\tableofcontents

\section{}

% REMOVED FROM electric_field_energy.ltx

We have also seen in various exercises that the Lorentz force could be obtained from the action

\begin{equation}\label{eqn:faradayLagrangian:20}
\begin{aligned}
S = \int \left(\inv{2} m \left(\frac{dx}{d\tau}\right)^2 + q A \cdot \frac{dx}{c d\tau}\right) d\tau
\end{aligned}
\end{equation}

and it appeared that this plus the Maxwell action

\begin{equation}\label{eqn:faradayLagrangian:40}
\begin{aligned}
S = \int \left(- \frac{c \epsilon_0}{2} (\grad \wedge A)^2 + J \cdot A \right) d^4 x
\end{aligned}
\end{equation}

as covered in \citep{classicalmechanics:PJMaxwellLagrangian} was required.  One action for the field
equation and one for the interaction equation.  Seeing the Lorentz force show up like this out of nowhere with only
manipulation of the Maxwell equation suggests that the Lorentz force or its associated Lagrangian is not actually that
fundamental.  We have one equation at the root of both (and that equation is probably quite close to the Maxwell field Lagrangian), probably
with the proper velocity \(mv^2/2\) term added in somehow.


%\bibliographystyle{plainnat}
%\bibliography{myrefs}

%\end{document}
