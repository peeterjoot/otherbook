%
% Copyright � 2012 Peeter Joot.  All Rights Reserved.
% Licenced as described in the file LICENSE under the root directory of this GIT repository.
%

%
%
%\documentclass{article}      % Specifies the document class

%\input{../peeters_macros.tex}

%
% The real thing:
%

%\usepackage[bookmarks=true]{hyperref}

\chapter{rindler}
\label{chap:rindler}
%\title{} % Declares the document's title.
%\author{Peeter Joot \quad peeterjoot@protonmail.com}         % Declares the author's name.
\date{ \(RCSfile: rindler.tex,v \) Last \(Revision: 1.7 \) \(Date: 2009/06/14 23:51:45 \) }
%
%\begin{document}             % End of preamble and beginning of text.

%\maketitle{}
%
%\tableofcontents
%
%\section{}

You had the following as the DE to solve with \(y = 1 + ax\)

\begin{equation}\label{eqn:rindler:20}
\begin{aligned}
y\ddot{y} &= a^2 - \dot{y}^2 \\
y\ddot{y} + \dot{y}^2 &= a^2 \\
\end{aligned}
\end{equation}

The solution appears to involve \(y\dot{y}\), which when differentiated
is

\begin{equation}\label{eqn:rindler:40}
\begin{aligned}
(y\dot{y})' &= y \ddot{y} + \dot{y}^2,
\end{aligned}
\end{equation}

as desired.  However, this can be integrated once more

\begin{equation}\label{eqn:rindler:60}
\begin{aligned}
y\dot{y} = \left(\inv{2} y^2\right)'.
\end{aligned}
\end{equation}

This puts your DE in the following convenient form

\begin{equation}\label{eqn:rindler:80}
\begin{aligned}
y\ddot{y} + \dot{y}^2 &= a^2 \\
&= \frac{d^2}{d\tau^2}\left(\inv{2} y^2\right).
\end{aligned}
\end{equation}

Integrating once, with constant of integration \(a \kappa\) gives

\begin{equation}\label{eqn:rindler:100}
\begin{aligned}
\frac{d}{d\tau}\left(\inv{2} y^2\right) &= a^2 \tau + a \kappa \\
\frac{d}{d\tau}\left(\inv{2} \left(1+ax\right)^2\right) &=  \\
\left(1+ax\right)\frac{d}{d\tau}\left(1+ax\right) &=  \\
\left(1+ax\right)a\frac{dx}{d\tau} &=  \\
\implies
\left(1+ax\right)\frac{dx}{d\tau} &= a\tau + \kappa \\
\end{aligned}
\end{equation}

Alternatively, integrating twice with \(\kappa = n/2\), and second integration constant \(a (m)/2\) we have

\begin{equation}\label{eqn:rindler:120}
\begin{aligned}
\inv{2} y^2 &= \inv{2} a^2 \tau^2 + \inv{2} a n \tau + \inv{2} a m \\
y^2 &= a^2 \tau^2 + a n\tau + a m \\
(1+ax)^2 &= \\
\implies
x
&= \sqrt{\tau^2 + n\tau + m} -1/a \\
&= \sqrt{\tau^2 + n\tau + (x_0 + 1/a)^2} -1/a \\
&= \sqrt{\tau^2 + (1 + a x_0)v_0 \tau + (x_0 + 1/a)^2} -1/a \\
\end{aligned}
\end{equation}

%\kappa = (1 + ax_0) v_0 = n/2

%\end{document}               % End of document.

