%
% Copyright � 2012 Peeter Joot.  All Rights Reserved.
% Licenced as described in the file LICENSE under the root directory of this GIT repository.
%

%
%
%\input{../peeter_prologue.tex}

%\chapter{Gradient and Divergence in different coordinate systems}
\label{chap:gradientAltCoord}
%\useCCL
\blogpage{http://sites.google.com/site/peeterjoot/math2009/gradientAltCoord.pdf}
\date{Aug 22, 2009}
\revisionInfo{\(RCSfile: gradientAltCoord.tex,v \) Last \(Revision: 1.7 \) \(Date: 2009/12/03 03:24:40 \)}

\beginArtWithToc
%\beginArtNoToc

\section{Motivation}

I wanted the Laplacian in spherical polar coordinates and figured that I ought to be able to figure it out, not look it up.  My recollection was that this could be calculated using a divergence theorem argument over a volume made small in the limit.  When I try this I get not the divergence, but the gradient.  Perhaps if I write up my notes more carefully I will see where I went wrong.

\section{Stokes setup}

The \R{3} divergence theorem was based on a curl dot product of a bivector with a volume element.  Starting there we want to write

\begin{equation}\label{eqn:curlDotVolumeDifferentialElement}
\begin{aligned}
(\grad \wedge B) \cdot d^3 x
\end{aligned}
\end{equation}

in terms of derivatives of \(B\).  Assuming the volume is parametrized by three differential displacements

\begin{equation}\label{eqn:gradientAltCoord:20}
\begin{aligned}
d\Bx_\alpha &= \PD{\alpha}{\Bx} d\alpha \\
d\Bx_\beta &= \PD{\beta}{\Bx} d\beta \\
d\Bx_\sigma &= \PD{\sigma}{\Bx} d\sigma
\end{aligned}
\end{equation}

Our volume element is the trivector

\begin{equation}\label{eqn:volumeElement}
\begin{aligned}
d^3 x \equiv d\Bx_\alpha  \wedge d\Bx_\beta  \wedge d\Bx_\sigma
\end{aligned}
\end{equation}

Assuming that we know the gradient in terms of an reciprocal pair of basis frames \(\gamma^\mu \cdot \gamma_\nu = \delta^\mu_\nu\) we write

\begin{equation}\label{eqn:gradientDefined}
\begin{aligned}
\grad \equiv \gamma^\mu \partial_\mu = \sum_\mu \gamma^\mu \PD{x^\mu}{}
\end{aligned}
\end{equation}

Now, an expansion of the differential element \eqnref{eqn:curlDotVolumeDifferentialElement} is possible.  Starting this we have

\begin{equation}\label{eqn:gradientAltCoord:40}
\begin{aligned}
(\grad \wedge B) \cdot d^3 x
&=
\gpgradezero{ (\grad \wedge B) d^3 x } \\
&=
\gpgradezero{ (\gamma^\mu \wedge \partial_\mu B) d^3 x } \\
&=
\inv{2} \gpgradezero{ (\gamma^\mu \partial_\mu B + \partial_\mu B \gamma^\mu) d^3 x } \\
&=
\inv{2} \gpgradezero{ \partial_\mu B (d^3 x \gamma^\mu + \gamma^\mu d^3 x) } \\
&=
\gpgradezero{ \partial_\mu B (d^3 x \cdot \gamma^\mu) } \\
&=
\partial_\mu B \cdot (d^3 x \cdot \gamma^\mu)
\end{aligned}
\end{equation}

The cyclic reordering property for the scalar product \(\gpgradezero{a b c} = \gpgradezero{ b c a}\) has been used as well as the vector-bivector wedge \(2 a \wedge B = a B + B a\), and trivector-vector dot product \(2 T \cdot a = T a + a T\).  Expanding this further we have

\begin{equation}\label{eqn:gradientAltCoord:60}
\begin{aligned}
(\grad \wedge B) \cdot d^3 x
&=
\partial_\mu B \cdot \left(
 (d\Bx_\alpha \wedge d\Bx_\beta) d\Bx_\sigma \cdot \gamma^\mu
-(d\Bx_\alpha \wedge d\Bx_\sigma) d\Bx_\beta \cdot \gamma^\mu
+(d\Bx_\beta  \wedge d\Bx_\sigma) d\Bx_\alpha \cdot \gamma^\mu
\right)
\end{aligned}
\end{equation}

Picking one of these terms as representative we have

\begin{equation}\label{eqn:gradientAltCoord:80}
\begin{aligned}
\partial_\mu B \cdot (d\Bx_\beta \wedge d\Bx_\sigma) d\Bx_\alpha \cdot \gamma^\mu
&=
\PD{x^\mu}{B} \cdot (d\Bx_\beta \wedge d\Bx_\sigma) \PD{\alpha}{x^\nu} \gamma_\nu \cdot \gamma^\mu d\alpha \\
&=
d\alpha \left( \PD{x^\mu}{B} \PD{\alpha}{x^\mu} \right) \cdot (d\Bx_\beta \wedge d\Bx_\sigma) \\
&=
d\alpha \PD{\alpha}{B} \cdot (d\Bx_\beta \wedge d\Bx_\sigma) \\
\end{aligned}
\end{equation}

The bivector curl differential element is thus reduced to

\begin{equation}\label{eqn:curDotReduction}
\begin{aligned}
(\grad \wedge B) \cdot d^3 x
&=
 d\alpha \PD{\alpha}{B} \cdot (d\Bx_\beta \wedge d\Bx_\sigma)
+d\beta \PD{\beta}{B} \cdot (d\Bx_\sigma \wedge d\Bx_\alpha)
+d\sigma \PD{\sigma}{B} \cdot (d\Bx_\alpha \wedge d\Bx_\beta)
\end{aligned}
\end{equation}

Although it is just algebra and chain rule getting us here, it is still impressive since it applies to arbitrary parametrization.  As derived there was also not even an assumption of orthogonal coordinates to express the gradient, nor any specific dimension or metric for the space (we could be considering a volume subspace parametrized by the differential displacements).

\section{Divergence theorem}

In \R{3} the Stokes differential element of \eqnref{eqn:curDotReduction} supplies us, via duality with the vector divergence theorem.  Writing \(B = -I a\), we have

\begin{equation}\label{eqn:gradientAltCoord:100}
\begin{aligned}
\grad \wedge B
&= -(\grad \wedge (I\Ba)) \\
&= - \gpgradethree{ \grad I\Ba } \\
&= - I (\grad \cdot \Ba )
\end{aligned}
\end{equation}

Factoring the pseudoscalar from the volume element \(d^3 x = I dV\) we have on the LHS

\begin{equation}\label{eqn:gradientAltCoord:120}
\begin{aligned}
(\grad \wedge B) \cdot d^3 x
&= - I (\grad \cdot \Ba ) I dV \\
&= (\grad \cdot \Ba ) dV \\
\end{aligned}
\end{equation}

Our terms on the RHS have the form

\begin{equation}\label{eqn:gradientAltCoord:140}
\begin{aligned}
d B \cdot d^2 x
&=
-(d I \Ba) \cdot d^2 x \\
&=
- \gpgradezero{ I d \Ba d^2 x } \\
&=
d \Ba \cdot (-I d^2 x)
\end{aligned}
\end{equation}

Assembling the results for this duality transformation we have the divergence theorem vector result in differential form

\begin{equation}\label{eqn:divergenceForm}
\begin{aligned}
(\grad \cdot \Ba ) dV &=
 d\alpha \PD{\alpha}{\Ba} \cdot (-I d\Bx_\beta \wedge d\Bx_\sigma)
+d\beta \PD{\beta}{\Ba} \cdot (-I d\Bx_\sigma \wedge d\Bx_\alpha)
+d\sigma \PD{\sigma}{\Ba} \cdot (-I d\Bx_\alpha \wedge d\Bx_\beta)
\end{aligned}
\end{equation}

\section{Examples (and faulty logic)}

\imageFigure{../figures/otherbook/polarAndSpherical}{Spherical and Cylindrical coordinate conventions}{fig:polarAndSpherical}{0.3}

\Cref{fig:polarAndSpherical} shows the coordinate unit vector conventions used in the below.  Following typical conventions a right handed pseudoscalar orientation has been picked.

\subsection{Cartesian coordinates}

With

\begin{equation}\label{eqn:gradientAltCoord:160}
\begin{aligned}
d\Bx &= dx \xcap \\
d\By &= dy \ycap \\
d\Bz &= dz \zcap
\end{aligned}
\end{equation}

and \(I = \xcap \ycap \zcap\), we have

\begin{equation}\label{eqn:gradientAltCoord:180}
\begin{aligned}
(\grad \cdot \Ba ) dx dy dz &=
 dx \PD{x}{\Ba} \cdot (-\xcap \ycap \zcap \ycap \zcap dy dz )  \\
&+dy \PD{y}{\Ba} \cdot (-\xcap \ycap \zcap \zcap \xcap dz dx )  \\
&+dz \PD{z}{\Ba} \cdot (-\xcap \ycap \zcap \xcap \ycap dx dy )
\end{aligned}
\end{equation}

Canceling \(dx dy dz\) on each side we have the unsurprising result

\begin{equation}\label{eqn:cartesianGradDot}
\begin{aligned}
\grad \cdot \Ba &=
\left(\xcap \PD{x}{}
+\ycap \PD{y}{}
+\zcap \PD{z}{} \right) \cdot \Ba
\end{aligned}
\end{equation}

\subsection{Cylindrical coordinates}

In cylindrical coordinates our differential elements are

\begin{equation}\label{eqn:gradientAltCoord:200}
\begin{aligned}
d\Br &= dx \rcap \\
d\Btheta &= r d\theta \thetacap \\
d\Bz &= dz \zcap
\end{aligned}
\end{equation}

with the pseudoscalar \(I = \rcap \thetacap \zcap\), we have

\begin{equation}\label{eqn:gradientAltCoord:220}
\begin{aligned}
(\grad \cdot \Ba ) r dr d\theta dz &=
 dr \PD{r}{\Ba} \cdot (-\rcap \thetacap \zcap \thetacap \zcap r d\theta dz )  \\
&+d\theta \PD{\theta}{\Ba} \cdot (-\rcap \thetacap \zcap \zcap \rcap dz dr )  \\
&+dz \PD{z}{\Ba} \cdot (-\rcap \thetacap \zcap \rcap \thetacap dr r d\theta )
\end{aligned}
\end{equation}

Again, canceling the volume elements we have

\begin{equation}\label{eqn:cylindricalGradDot}
\begin{aligned}
\grad \cdot \Ba &=
\left(\rcap \PD{r}{}
+\thetacap \inv{r} \PD{\theta}{}
+\zcap \PD{z}{} \right) \cdot \Ba
\end{aligned}
\end{equation}

\subsection{Spherical coordinates}

Last of interest is the spherical coordinates where our differential elements are

\begin{equation}\label{eqn:gradientAltCoord:240}
\begin{aligned}
d\Br &= dr \rcap \\
d\Btheta &= r d\theta \thetacap \\
d\Bphi &= d\phi r \sin\theta \phicap
\end{aligned}
\end{equation}

with the pseudoscalar \(I = \rcap \phicap \thetacap\), we have

\begin{equation}\label{eqn:gradientAltCoord:260}
\begin{aligned}
(\grad \cdot \Ba ) r^2 \sin\theta dr d\phi d\theta &=
 dr \PD{r}{\Ba} \cdot (-\rcap \phicap \thetacap \phicap \thetacap r^2 \sin\theta d\phi d\theta )  \\
&+d\phi \PD{\phi}{\Ba} \cdot (-\rcap \phicap \thetacap \thetacap \rcap r d\theta dr )  \\
&+d\theta \PD{\theta}{\Ba} \cdot (-\rcap \phicap \thetacap \rcap \phicap dr r \sin\theta d\phi )
\end{aligned}
\end{equation}

Again, canceling the volume elements we have

\begin{equation}\label{eqn:sphericalGradDot}
\begin{aligned}
\grad \cdot \Ba &=
\left(\rcap \PD{r}{}
+\phicap \inv{r \sin\theta} \PD{\phi}{}
+\thetacap \inv{r} \PD{\theta}{}
\right) \cdot \Ba
\end{aligned}
\end{equation}

\subsection{What is wrong above?}

Looking at this anew, the cancellation of the volume elements is fishy seeming.  It has actually produced the correct results for the gradient in each of these coordinate systems, but for anything but the most infinitesimal volume, the area elements in the differential elements are also going to be functions that vary.  So, while it appears we can use this to blunder upon the correct results for the gradient in each of our example coordinate systems, this abusive mathematics will not give us the divergence expression that is desired.  We have however found the gradient in each of the Cartesian, cylindrical, and spherical coordinate systems, which in summary are

\begin{equation}\label{eqn:gradientExampleSummary}
\begin{aligned}
\grad
&= \xcap \PD{x}{} +\ycap \PD{y}{} +\zcap \PD{z}{} \\
&= \rcap \PD{r}{} +\thetacap \inv{r} \PD{\theta}{} + \zcap \PD{z}{} \\
&= \rcap \PD{r}{} +\phicap \inv{r \sin\theta} \PD{\phi}{} +\thetacap \inv{r} \PD{\theta}{}
\end{aligned}
\end{equation}

To consider the divergence itself we will have to be more careful.

\subsection{General orthonormal curvilinear coordinates}

Before moving on to the divergence, let us consider a general orthonormal curvilinear frame.  As conventional we write

\begin{equation}\label{eqn:gradientAltCoord:280}
\begin{aligned}
d\Bx_1 &= h_1 dx_1 \Be_1 \\
d\Bx_2 &= h_2 dx_2 \Be_2 \\
d\Bx_3 &= h_3 dx_3 \Be_3
\end{aligned}
\end{equation}

and \(I = \Be_1 \Be_2 \Be_3\).  The volume cancellation now gives us

\begin{equation}\label{eqn:curvalinearVolumeCancellation}
\begin{aligned}
\grad \cdot \Ba &=
\inv{h_1 h_2 h_3} \left(
  \PD{x_1}{\Ba} \cdot (\Be_1 h_2 h_3 )
+ \PD{x_2}{\Ba} \cdot (\Be_2 h_3 h_1 )
+ \PD{x_3}{\Ba} \cdot (\Be_3 h_1 h_2 )  \right)
\end{aligned}
\end{equation}

So the gradient is left as

\begin{equation}\label{eqn:gradientCurva}
\begin{aligned}
\grad &=
 \Be_1 \inv{h_1} \PD{x_1}{}
+\Be_2 \inv{h_2} \PD{x_2}{}
+\Be_3 \inv{h_3} \PD{x_3}{}
\end{aligned}
\end{equation}

Here the unit vectors \(\Be_k\) and the functions \(h_k\) are functions of the parametrization.

\section{Divergence}

\subsection{The problem}

A more careful evaluation of what led to \eqnref{eqn:curvalinearVolumeCancellation} is expected to supply us with the generalized divergence representation.  We want to consider the limit

\begin{equation}\label{eqn:gradientAltCoord:300}
\begin{aligned}
\grad \cdot \Ba &=
\inv{h_1 h_2 h_3}
\lim_{
\Delta x_1, \Delta x_2, \Delta x_3 \rightarrow 0}
\inv{\Delta x_1 \Delta x_2 \Delta x_3
} \\
&\iiint dx_1 dx_2 dx_3 \left(
h_2 h_3 \Be_1 \PD{x_1}{}
+h_3 h_1 \Be_2 \PD{x_2}{}
+h_1 h_2 \Be_3 \PD{x_3}{} \right) \cdot \Ba
\end{aligned}
\end{equation}

\subsection{A simpler problem}

To gain some clarity, let us divide and conquer.  Defer the more complex 3D divergence problem above temporarily, instead considering the simpler but similar two parameter problem.  First required is the differential form for this curl surface integral.  A super speed derivation of this form from the curl-area dot product is

\begin{equation}\label{eqn:gradientAltCoord:320}
\begin{aligned}
(\grad \wedge \Ba) \cdot d^2 x
&=
\gpgradezero{ (\gamma^m \wedge \partial_m \Ba) d^2 x} \\
&=
\inv{2} \gpgradezero{ \gamma^m (\partial_m \Ba) d^2 x - (\partial_m \Ba) \gamma^m d^2 x} \\
&=
(\partial_m \Ba) \cdot \inv{2} \gpgradeone{ d^2 x \gamma^m - \gamma^m d^2 x} \\
&=
(\partial_m \Ba) \cdot ( d^2 x \cdot \gamma^m ) \\
&=
(\partial_m \Ba) \cdot ( d\Bx_\alpha d\Bx_\beta \cdot \gamma^m - d\Bx_\beta d\Bx_\alpha \cdot \gamma^m) \\
&=
d\beta \PD{\beta}{\Ba} \cdot d\Bx_\alpha -d\alpha \PD{\alpha}{\Ba} \cdot d\Bx_\beta
\end{aligned}
\end{equation}

\imageFigure{../figures/otherbook/parameterizedSurface}{Two parameter surface}{fig:parameterizedSurface}{0.3}

For the orthonormal curvilinear coordinate problem, we can write \(\PDi{\alpha}{\Bx} = h_1 \Be_1\), and \(\PDi{\beta}{\Bx} = h_2 \Be_2\), and are left with a requirement to evaluate

\begin{equation}\label{eqn:gradientAltCoord:340}
\begin{aligned}
(\grad \wedge \Ba) \cdot d^2 x
&=
d\beta \PD{\beta}{\Ba} \cdot \Be_1 h_1 d\alpha -d\alpha \PD{\alpha}{\Ba} \cdot \Be_2 h_2 d\beta
\end{aligned}
\end{equation}

Our problem is now reduced to evaluation of a scalar valued integrals of the form

\begin{equation}\label{eqn:gradientAltCoord:360}
\begin{aligned}
I = \iint \left( d\alpha \PD{\alpha}{\Ba} \right) \cdot \left( d\beta \PD{\beta}{\Bx} \right)
\end{aligned}
\end{equation}

over a vector parametrized \(\Bx = \Bx(\alpha, \beta)\) surface, as depicted in \cref{fig:parameterizedSurface}.  This has reduced the problem to evaluation of scalar double integrals of the form

\begin{equation}\label{eqn:gradientAltCoord:380}
\begin{aligned}
I = \iint h \PD{\alpha}{a} d\alpha d\beta
\end{aligned}
\end{equation}

To make this more real, consider the specific example of polar form coordinates in a plane.  We then have

\begin{equation}\label{eqn:gradientAltCoord:400}
\begin{aligned}
(\grad \wedge \Ba) \cdot (\rcap \wedge \thetacap r dr d\theta)
&=
d\theta \PD{\theta}{\Ba} \cdot \rcap dr -dr \PD{r}{\Ba} \cdot \thetacap r d\theta \\
&=
dr d\theta \left( \PD{\theta}{a_r} - r \PD{r}{a_\theta} \right)
\end{aligned}
\end{equation}

%\EndArticle
\EndNoBibArticle
