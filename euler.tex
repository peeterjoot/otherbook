%
% Copyright � 2012 Peeter Joot.  All Rights Reserved.
% Licenced as described in the file LICENSE under the root directory of this GIT repository.
%

%
%
%\documentclass{article}

%\input{../peeters_macros.tex}
%\input{../peeters_macros2.tex}

%\usepackage{listings}
%\usepackage{txfonts} % for ointctr... (also appears to make "prettier" \int and \sum's)
%\usepackage[bookmarks=true]{hyperref}

%\usepackage{color,cite,graphicx}
   % use colour in the document, put your citations as [1-4]
   % rather than [1,2,3,4] (it looks nicer, and the extended LaTeX2e
   % graphics package.
%\usepackage{latexsym,amssymb,epsf} % do not remember if these are
   % needed, but their inclusion can not do any damage


\chapter{euler}
\label{chap:euler}
%\author{Peeter Joot \quad peeterjoot@protonmail.com }
\date{ March dd, 2009.  \(RCSfile: euler.tex,v \) Last \(Revision: 1.5 \) \(Date: 2009/06/14 23:51:45 \) }

%\begin{document}

%\maketitle{}
%
%\tableofcontents
%
%\section{}

I do not know if you have seen it, but Feynman's lectures (volume II in what he calls the "entertainment" chapter)
has some nice simple examples of these ideas.
They cut the complexity out of the picture and treat some simple concrete cases, like the following

\begin{equation}\label{eqn:euler:20}
\begin{aligned}
S = \int_{a}^b \dot{y}^2 dx
\end{aligned}
\end{equation}

Let

\begin{equation}\label{eqn:euler:40}
\begin{aligned}
y = \overbar{y} + \alpha
\end{aligned}
\end{equation}

Here \(\overbar{y}\) is the solution that you are looking for, and \(\alpha(a) = \alpha(b) = 0\) (vanishes at the end points).

Taking derivatives
\begin{equation}\label{eqn:euler:60}
\begin{aligned}
\dot{y} &= \dot{\overbar{y}} + \dot{\alpha}
\end{aligned}
\end{equation}

\begin{equation}\label{eqn:euler:80}
\begin{aligned}
S
&= \int_{a}^b \left(\dot{\overbar{y}} + \dot{\alpha} \right)^2 dx \\
&= \int_{a}^b \left(\dot{\overbar{y}}^2 + 2 \dot{\alpha}\dot{\overbar{y}} + \dot{\alpha}^2  \right) dx \\
&= \int_{a}^b \left(\overbar{\dot{y}}^2 - 2 {\alpha}\ddot{\overbar{y}} - {\alpha}\ddot{\alpha}  \right) dx \\
\end{aligned}
\end{equation}

and for the derivative to be zero
\begin{equation}\label{eqn:euler:100}
\begin{aligned}
0 &= \left. \frac{dS }{d\alpha} \right\vert_{\alpha = 0} \\
&= -2 \int_{a}^b \ddot{\overbar{y}} dx \\
\end{aligned}
\end{equation}

Dropping the overbar, you have for the desired solution to the variation

\begin{equation}\label{eqn:euler:120}
\begin{aligned}
\ddot{y} = 0
\end{aligned}
\end{equation}

(particle moves with constant velocity in absence of force).

%\end{document}
