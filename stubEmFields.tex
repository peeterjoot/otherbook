%
% Copyright � 2012 Peeter Joot.  All Rights Reserved.
% Licenced as described in the file LICENSE under the root directory of this GIT repository.
%

%
%
%\documentclass{article}

%\input{../peeters_macros.tex}
%\input{../peeters_macros2.tex}

%\usepackage[bookmarks=true]{hyperref}

%\usepackage{color,cite,graphicx}
   % use colour in the document, put your citations as [1-4]
   % rather than [1,2,3,4] (it looks nicer, and the extended LaTeX2e
   % graphics package.
%\usepackage{latexsym,amssymb,epsf} % do not remember if these are
   % needed, but their inclusion can not do any damage

\chapter{stub em fields}
\label{chap:stubEmFields}
%\author{Peeter Joot \quad peeterjoot@protonmail.com}
\date{ Feb dd, 2009.  \(RCSfile: stubEmFields.tex,v \) Last \(Revision: 1.9 \) \(Date: 2009/06/14 23:51:45 \) }

%\begin{document}

%\maketitle{}

%\tableofcontents

\section{}

\section{Misc.  Does not fit yet}

Taken from \citep{gabookII:PJFourierVacuum}


\subsection{Separate electric and magnetic fields with the boundary conditions unspecified}

Fixing the boundary value in terms of the initial field at \(t=0\) is not the only option.  We see similar things in classical mechanics in constant acceleration problems where one can use an initial position and velocity, or positions at two different times, and so forth.  Bohm leaves his equivalents to the integration constants \(C_{\Bk}\) unspecified.  If we do so too we have \eqnref{eqn:stub_em_fields:undetermined}.  How does that look with separate fields?

FIXME:TODO:...

Cut and Paste Note.  That referenced equation was:
\begin{align}\label{eqn:stub_em_fields:undetermined}
F(\Bx,t) = \sum_{\Bk}
\exp\left(i \Bk c t \right)
\exp\left(-i \Bk \cdot \Bx \right)
C_{\Bk}
\end{align}

%\bibliographystyle{plainnat}
%\bibliography{myrefs}

%\end{document}
