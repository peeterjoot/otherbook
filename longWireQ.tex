%
% Copyright � 2012 Peeter Joot.  All Rights Reserved.
% Licenced as described in the file LICENSE under the root directory of this GIT repository.
%

%
%
%\documentclass{article}      % Specifies the document class

%\input{../peeters_macros.tex}

%\usepackage[bookmarks=true]{hyperref}

                             % The preamble begins here.
%\chapter{long wire q} % Declares the document's title.
\label{chap:longWireQ}
%\author{Peeter Joot \quad peeterjoot@protonmail.com}         % Declares the author's name.
\date{ \(RCSfile: longWireQ.tex,v \) Last \(Revision: 1.7 \) \(Date: 2009/06/14 23:51:45 \) }

%\begin{document}             % End of preamble and beginning of text.

%\maketitle{}
%\tableofcontents
%
%\section{}

Magnetostatics field:

\begin{equation*}
F = E + icB = icB
\end{equation*}

Or,
\begin{equation*}
iF = -cB
\end{equation*}

(here I am using Doran/Lasenby notation writing \(E\) and \(B\) in bivector form: \(E = E^j \sigma_j = E^j \gamma_j \wedge \gamma_0\)).

\begin{equation}\label{eqn:longWireQ:20}
\begin{aligned}
\grad F
&= \grad icB \\
&= -i c \grad B \\
&= -i c \left( \grad \cdot B + \grad \wedge B \right) \\
&= J/\epsilon_0 c
\end{aligned}
\end{equation}

The multiplication by \(i\) turns the grade one (dot) and three (wedge) parts into grade three and one respectively.  Equating grades on the left and right of the equation one has:

\begin{equation*}
0 = -i c \grad \cdot B = i \grad \cdot ( iF )
\end{equation*}

and
\begin{equation*}
J/\epsilon_0 c = -i c \grad \wedge B = i \grad \wedge (iF)
\end{equation*}

Assuming equation 16, integration gives

\begin{equation}\label{eqn:longWireQ:40}
\begin{aligned}
\int_S i \grad \wedge (iF)
&= i \int_{\partial S} iF \\
&= -\int_{\partial S} F \\
&= \int_S J/\epsilon_0 c.
\end{aligned}
\end{equation}

Or
\begin{equation*}
\int_{\partial S} F = -I_S/\epsilon_0 c.
\end{equation*}

%\end{document}               % End of document.
