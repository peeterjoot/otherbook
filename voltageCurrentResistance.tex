%
% Copyright � 2012 Peeter Joot.  All Rights Reserved.
% Licenced as described in the file LICENSE under the root directory of this GIT repository.
%

%
%
%\documentclass{article}

%\input{../peeters_macros.tex}
%\input{../peeters_macros2.tex}

%\usepackage{listings}
%\usepackage{txfonts} % for ointctr... (also appears to make "prettier" \int and \sum's)
% makes \grad look funny though (almost like spacegrad, but narrower)
%\usepackage[bookmarks=true]{hyperref}

%\usepackage{color,cite,graphicx}
   % use colour in the document, put your citations as [1-4]
   % rather than [1,2,3,4] (it looks nicer, and the extended LaTeX2e
   % graphics package.
%\usepackage{latexsym,amssymb,epsf} % do not remember if these are
   % needed, but their inclusion can not do any damage


\chapter{Voltage current, and resistance}
\label{chap:voltageCurrentResistance}
%\author{Peeter Joot \quad peeterjoot@protonmail.com }
\date{ May 3, 2009.  \(RCSfile: voltageCurrentResistance.tex,v \) Last \(Revision: 1.7 \) \(Date: 2009/06/14 23:51:45 \) }

%\begin{document}

%\maketitle{}
%\tableofcontents
%\section{}

Hi Dad,

By what you have said, I am guessing that you have encountered the V I R ``triangle'' where any one can be ``defined'' in terms of the other two.
These are not definitions, but are really
just a pictorial description of the following formula, which can be written in three different ways all logically identical

\begin{equation}\label{eqn:voltageCurrentResistance:20}
\begin{aligned}
V &= I R \\
I &= \frac{V}{R} \\
R &= \frac{V}{I} \\
\end{aligned}
\end{equation}

As you have said this is very cyclic, because none of these are definitions.

To make things worse, if you go looking for descriptions of what each of these are, what you find may very well not even be true.
Here is an example that I just found on the Internet, and it is not terribly different from descriptions that I had seen in various
``hobbyist electronics explained'' books.

\begin{lstlisting}
Voltage is the electrical force, or "pressure",
that causes current to flow in a circuit.
It is measured in VOLTS (V or E).
Take a look at the diagram.  Voltage would be
the force that is pushing the water (electrons) forward.
\end{lstlisting}

Descriptions like this make it very hard to move from electronics hobbyist mode to physics mode.  Even an introductory
physics course (like my first grade 12 physics course in high school for example) will initially be confusing with this sort of description because voltage is not a force.  You have to unlearn this sort of stuff above before things start to make sense.

Mathematically, voltage times charge (the product of the two) is a measure of energy, not force.
Similarly, energy has the mathematical description as force times distance.  You can keep going with these and
eventually you will find that you need a few fundamental definitions to define everything else related to particle motion
and electromagnetism.  These are

\begin{enumerate}
\item mass
\item distance
\item time
\item charge
\end{enumerate}

Part of the problem with the fact that misleading descriptions like the above exist, is that to do it
properly, there is a small hierarchy of terminology that is required.  You need a bit
of prerequisite nomenclature to get to writing a sensible description of voltage and friends, as in
the \(V = I R\) relationship.  Here is what you need
before you can get to voltage

\begin{equation}\label{eqn:voltageCurrentResistance:40}
\begin{aligned}
\text{velocity} &= \text{distance}/ \text{time} \\
\text{acceleration} &= \text{velocity} / \text{time} \\
\text{force} &= \text{mass} \times \text{acceleration} \\
\text{energy} &= \text{force} \times \text{distance}
\end{aligned}
\end{equation}

(strictly speaking all but energy above are directed quantities, and I had incur the wrath of my grade twelve teacher for
 writing the above without qualification).

With these you really only need velocity to define the current, and one can write

\begin{equation}\label{eqn:voltageCurrentResistance:60}
\begin{aligned}
\text{current} (I) = \text{velocity} \times \text{charge}
\end{aligned}
\end{equation}

However, voltage is actually a fairly more abstract quantity and you need a bit more to define it.
In particular, you can then define voltage in terms of energy (which is already an abstract quantity compared to
the concrete quantities like mass, distance, charge and time).

Energy is required
to move towards or separate (if they the charges are attractive) any two (or more) charged objects.
Saying that energy is required means that work must be done, where a force is exerted over a distance to move repulsive
charges towards each other into some specific configuration.  If two like charges, two electrons say, are far enough away
from each other that there is not a significant or easily measurable force between them, and you move them towards each other
to where a force between them can be measured, then one can say there is an energy bill associated with the final configuration
of the charges.

You do not necessarily have to know the route that the charges got there.  If one of those ``charges'' is a
cat, and you have walked that cat around the room towards a balloon on a table seven times before getting it near enough
to the balloon that the cat hair is sticking out towards the balloon, then the net electrical effect on the cat hair will
be the same as if you have walked that cat straight towards the balloon.  The voltage is a measure of the energy bill associated
with a charge configuration.  Now here the cat analogy breaks down a bit because it does not take much energy to walk that cat
away from the balloon (the attractive force between the cat hair and the balloon just is not that big), but there is charge
all over that balloon, and charge all over the cat's hair, and the energy per cat required to walk that cat away from the balloon
is what one would call the voltage.

More exactly one could define this voltage indirectly in terms of energy

\begin{equation}\label{eqn:voltageCurrentResistance:80}
\begin{aligned}
\text{energy} = \text{charge} \times \text{voltage}
\end{aligned}
\end{equation}

or, equivalently
\begin{equation}\label{eqn:voltageCurrentResistance:100}
\begin{aligned}
\text{voltage} = \text{energy} / \text{charge}
\end{aligned}
\end{equation}

Here, voltage is really the voltage measured for the cat and balloon charge configuration, and the energy is
the energy required to counteract the electrical attraction between the cat hair and the balloon when they are close
compared to when they are far enough away that there is no longer much attraction.  In the above the charge is really
the charge of the cat ... this is the charge used to 'probe' the charge configuration of the balloon.

Notice that you can not really define voltage as an absolute quantity.  There is always two spatially separated points
involved (ie: the close distance between the cat and balloon and the far distance).  You can just as easily define a
voltage between twill not so far separated points in space, and the energy in this case is just the energy to move
that distance.

To properly define the voltage for the energy bill associated with walking the cat towards the bill, you have to have
a careful accountant.   Specifically, he has to be instructed to only count the energy required or gained that was
associated with moving the cat to or away from the balloon.  It will take more work for you to personally walk around the
room seven times, but this work is muscular work and does not have much to do with the balloon's electrical properties.
If you did not want to count that walking in the work, you can start talking about tossing cats in outer space towards or
away from the balloons, but let us say instead that you have a careful accountant.

Now, I do not know if that really ended up being a good description of voltage or not, but if it is, you can then define
resistance in terms of current and voltage and it will not be at all cyclic since both current and voltage have been defined
in terms of the fundamental quantities (time, mass, distance, charge).

%\end{document}
