%
% Copyright � 2012 Peeter Joot.  All Rights Reserved.
% Licenced as described in the file LICENSE under the root directory of this GIT repository.
%

%
%
%\documentclass{article}

%\input{../peeters_macros.tex}
%\input{../peeters_macros2.tex}

%\usepackage[bookmarks=true]{hyperref}

%\usepackage{color,cite,graphicx}
   % use colour in the document, put your citations as [1-4]
   % rather than [1,2,3,4] (it looks nicer, and the extended LaTeX2e
   % graphics package.
%\usepackage{latexsym,amssymb,epsf} % do not remember if these are
   % needed, but their inclusion can not do any damage


\chapter{Composition of rotations exercise.  Two nineties}
\label{chap:twoNinetyRotations}
%\author{Peeter Joot \quad peeterjoot@protonmail.com}
\date{ Jan 17, 2009.  \(RCSfile: twoNinetyRotations.tex,v \) Last \(Revision: 1.8 \) \(Date: 2009/06/14 23:51:45 \) }

%\begin{document}

%\maketitle{}
%\tableofcontents
\section{Problem}

Rotate 90 about the z-axis, and then 90 about the new x-axis (problem from Alan M's book draft).

\section{Solution}

The z-axis rotation is

\begin{equation}\label{eqn:twoNinetyRotations:20}
\begin{aligned}
R_{z,90}(\Bx) &= e^{-\Be_{12}} \Bx e^{\Be_{12}}
\end{aligned}
\end{equation}

and the rotation about the new x-axis (ie: in the old -1,3 plane) is

\begin{equation}\label{eqn:twoNinetyRotations:40}
\begin{aligned}
R_{x',90}(\Bx') &= e^{\Be_{13}} \Bx' e^{-\Be_{13}}
\end{aligned}
\end{equation}

Therefore the composite rotation is

\begin{equation}\label{eqn:twoNinetyRotations:60}
\begin{aligned}
R(\Bx) &= e^{\Be_{13}} e^{-\Be_{12}} \Bx e^{\Be_{12}} e^{-\Be_{13}}
\end{aligned}
\end{equation}

We want to expand the product

\begin{equation}\label{eqn:twoNinetyRotations:80}
\begin{aligned}
R
&= e^{\Be_{13}} e^{-\Be_{12}} \\
&= \inv{2} (1 + \Be_{13}) (1 - \Be_{12}) \\
&= \inv{2} (\Be_1 - \Be_{3}) \Be_1 \Be_1 (\Be_1 - \Be_{2}) \\
&= \inv{2} (\Be_1 - \Be_{3}) \cdot (\Be_1 - \Be_{2}) +\inv{2} (\Be_1 - \Be_{3}) \wedge (\Be_1 - \Be_{2}) \\
&= \inv{2} + \frac{\sqrt{3}}{2} \frac{(\Be_1 - \Be_{3}) \wedge (\Be_1 - \Be_{2})}{\sqrt{3}} \\
\end{aligned}
\end{equation}

Letting \(i = ((\Be_1 - \Be_{2}) \wedge (\Be_1 - \Be_{3}))/\sqrt{3}\) we have

\begin{equation}\label{eqn:twoNinetyRotations:100}
\begin{aligned}
R
&= \cos(\pi/3) - i\sin(\pi/3) \\
&= e^{-i \pi/3}
\end{aligned}
\end{equation}

So, the composite rotation will take vectors that lie in the \((\Be_1 - \Be_{2}) \wedge (\Be_1 - \Be_{3})\) plane, and rotate them by \(2\pi/3 = 120^\circ\).

In terms of a normal we can write the plane in its dual form

\begin{equation}\label{eqn:twoNinetyRotations:120}
\begin{aligned}
i &= \tilde{I}(I i) = -I \Bn
\end{aligned}
\end{equation}

So the normal of the rotational plane is

\begin{equation}\label{eqn:twoNinetyRotations:140}
\begin{aligned}
\Bn
&= \inv{\sqrt{3}} \Be_{123} \left( -\Be_{13} - \Be_{21} + \Be_{23} \right) \\
&= \frac{-1}{\sqrt{3}} \left( \Be_{2} + \Be_{3} + \Be_{1} \right) \\
\end{aligned}
\end{equation}

So we can also write this rotation as a rotation about the \(\Be_1 + \Be_2 + \Be_3\) axis (with a sense that I had have to think about to get right).

%\bibliographystyle{plainnat}
%\bibliography{myrefs}

%\end{document}
