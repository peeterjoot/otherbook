%
% Copyright � 2012 Peeter Joot.  All Rights Reserved.
% Licenced as described in the file LICENSE under the root directory of this GIT repository.
%

%
%
%\documentclass{article}

%\input{../peeters_macros.tex}
%\input{../peeters_macros2.tex}

%\usepackage[bookmarks=true]{hyperref}

%\chapter{review alan ch4}
\label{chap:reviewAlanGabookCh4}
%\author{Peeter Joot \quad peeterjoot@protonmail.com}
\date{ Mmm dd, 2009.  \(RCSfile: reviewAlanGabookCh4.tex,v \) Last \(Revision: 1.7 \) \(Date: 2009/06/14 23:51:45 \) }

%\begin{document}

%\maketitle{}
%\tableofcontents

\section{Chapter 4. Inner product spaces}

\subsection{page 48}

You have:

"We often write \(\inv{\Abs{\Bu}}\Bu = \frac{\Bu}{\Abs{\Bu}}\)"

did you mean to say

"We often write \(\ucap = \frac{\Bu}{\Abs{\Bu}}\)"

?

\subsection{page 49}

In high school I recall wondering why we used \(\Be_i\) for unit vectors, instead of \(\Bu_i\) say.
I
believe its for one-length vector (ein in German).  Perhaps silly
, but I had love to have seen this in my intro linear algebra book (take the mystery out of the symbols.)

\section{Chapter 5.  Geometric algebra}

\subsection{page 65}

last paragraph:

  "We can only compare orientations"

would make more sense to say

  "We can only compare orientations and total area"

(thinking of transported vectors as an analogy.  you can compare parallel vectors with different magnitudes).

\subsection{page 67}

typo, last line.  trivector should be bivector.

\subsection{page 68}

Figure 5.4.  I do not think the orientation displayed does not make sense as depicted.  Better would be to butt up the tail of \(\Bv\) with the head of \(\Bu\).

Theorem 5.2.  With the content of the text up to this point this seems like a
definition and not a theorem to me, and is somewhat cyclic.  ie: A quantity

\begin{equation}\label{eqn:reviewAlanGabookCh4:20}
\begin{aligned}
\Bu \wedge \Bv \equiv \Abs{\Bu} \Abs{\Bv} \sin \theta (\Be_1 \wedge \Be_2)
\end{aligned}
\end{equation}

can be said to represent the oriented area of the parallelogram spanned by the
vectors \(\Bu\) and \(\Bv\).  This however gets you into trouble conceptually
since you have not really defined what \(\Be_1 \wedge \Be_2\) means.

Conceptually speaking I think that page 68 and 69 need to be reworked, but
doing it well is tricky since you have conflicting goals of introducing some
intuition as well as being rigorous.  As a minimum I think that Theorem 5.3
O2, O3, O4 should be the fundamental definition that you work from (with O1,
and 5.2 as consequences).  From 5.2 I think that you can then motivate a
bivector norm definition based on the magnitude of the parallelogram.

\subsection{page 71}

Figure 5.8.  Like the bivector, if you put your arrows head to tail (ie: \(\Bv\), and \(\Bu\) transported parallel to the top face) the orientation arrows that
you have made would be clearer.

Also, similar to the preceding bivector page, your outer product definition is
fuzzy, based on intuition and comparison to the described trivector.  Better
IMO would be an axiomatic definition that builds on a two vector outer product definition:

\begin{equation}\label{eqn:reviewAlanGabookCh4:40}
\begin{aligned}
\Bu \wedge \Bv \wedge \Bw &= (\Bu \wedge \Bv) \wedge \Bw = \Bu \wedge (\Bv \wedge \Bw) \\
\end{aligned}
\end{equation}

\subsection{page 74}

Your "click" does not work for me. What is it supposed to do?

\subsection{page 75}

You could connect this complex number bit this with your "Wait!" mixed bag discussion back on page 72, since this
shows that the the apples plus oranges addition is not actually that unfamiliar.

\subsection{page 81}

description of the Feynman figure looks wrong.  To me it looks like 360, 480, 720.  You have 180, 360, 720.

I tried this and I can not get my arm twisted that last 180.

\subsection{page 82}

Ex. 5.20.  Text description I think would be easier to understand if you said for the second rotation that you are to
rotate about the new x-axis (instead of x-axis).

Now, for this problem when I work it, I get a different answer, so we should
compare notes.  \href{http://sites.google.com/site/peeterjoot/math2009/two_ninety_rotations.pdf}{I have put mine on my website.}.  One of us has a mistake, or I am misinterpreting your problem.

%\subsection{}
%\subsection{}
%\subsection{}

%\bibliographystyle{plainnat}
%\bibliography{myrefs}

%\end{document}
