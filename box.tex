%
% Copyright � 2012 Peeter Joot.  All Rights Reserved.
% Licenced as described in the file LICENSE under the root directory of this GIT repository.
%

%
%
%\documentclass{article}

%\input{../peeters_macros.tex}
%\input{../peeters_macros2.tex}

%\usepackage[bookmarks=true]{hyperref}

%\usepackage{color,cite,graphicx}
   % use colour in the document, put your citations as [1-4]
   % rather than [1,2,3,4] (it looks nicer, and the extended LaTeX2e
   % graphics package.
%\usepackage{latexsym,amssymb,epsf} % do not remember if these are
   % needed, but their inclusion can not do any damage


\chapter{box}
\label{chap:box}
%\author{Peeter Joot \quad peeterjoot@protonmail.com}
\date{ Mmm dd, 2008.  \(RCSfile: box.tex,v \) Last \(Revision: 1.7 \) \(Date: 2009/06/14 23:51:45 \) }

%\begin{document}

%\maketitle{}
%
%\tableofcontents
%\section{}


\begin{equation}\label{eqn:box:20}
\begin{aligned}
(\gamma^\mu \partial_\mu)^2
&=
(\gamma^\mu \partial_\mu) \cdot (\gamma^\nu \partial_\nu)
+ (\gamma^\mu \partial_\mu) \wedge (\gamma^\nu \partial_\nu) \\
&= (\gamma^\mu \partial_\mu) \cdot (\gamma_\nu \partial^\nu) \\
&= \gamma^\mu \cdot \gamma_\nu \partial_\mu \partial^\nu \\
&= {\delta^\mu}_\nu \partial_\mu \partial^\nu \\
&= \partial_\mu \partial^\mu \\
\end{aligned}
\end{equation}


\begin{equation}\label{eqn:box:40}
\begin{aligned}
\Box
&= \eta ^{\mu\nu}\partial_{\nu}  \partial_{\mu}  \\
&= \frac{1}{2}\gamma^{\mu}\gamma^{\nu} \partial_{\nu}  \partial_{\mu} +\frac{1}{2}\gamma^{\nu}\gamma^{\mu} \partial_{\nu}  \partial_{\mu} \\
&= \frac{1}{2}\left(\gamma^{\mu}\gamma_{\nu} + \gamma_{\nu}\gamma^{\mu} \right) \partial^{\nu} \partial_{\mu} \\
&= \gamma^\mu \cdot \gamma_\nu \partial^{\nu} \partial_{\mu} \\
&= \partial^{\mu} \partial_{\mu} \\
\end{aligned}
\end{equation}


Avodyne:

Here is how the Dirac square-root thing works in a more pedestrian notation.  First we write \(\gamma^\mu\gamma^\nu\) as the sum of a commutator and an anticommutator:

\begin{equation}\label{eqn:box:60}
\begin{aligned}
\gamma^\mu \gamma^\nu = \inv{2}[\gamma^\mu,\gamma^\nu] + \inv{2} \{\gamma^\mu,\gamma^\nu\}
\end{aligned}
\end{equation}

The first term is antisymmetric on exchange of \(\mu\) and \(\nu\), while the second term is symmetric.  Also, partial derivatives commute, so \(\partial_\mu\partial_\nu\) is symmetric.  Then, in general, if you contract both indices of an antisymmetric tensor \(A^{\mu\nu}\) with those of a symmetric tensor \(S_{\mu\nu}\), you get zero: \(A^{\mu\nu}S_{\mu\nu}=0\).
So, the commutator term vanishes when contracted.  And the anticommutator is

\begin{equation}\label{eqn:box:80}
\begin{aligned}
\inv{2}\{\gamma^\mu,\gamma^\nu\}=\eta^{\mu\nu}
\end{aligned}
\end{equation}

%\bibliographystyle{plainnat}
%\bibliography{myrefs}

%\end{document}
